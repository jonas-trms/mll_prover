\documentclass{article}
\usepackage{graphicx} % Required for inserting images
\usepackage{amssymb,cmll}
\usepackage{ebproof}

%imported from click and collect
\newcommand*{\orth}{^\perp}
\newcommand*{\tensor}{\otimes}
\newcommand*{\one}{1}
\newcommand*{\plus}{\oplus}
\newcommand*{\zero}{0}
\newcommand*{\limp}{\multimap}

\newcommand*{\hypv}[1]{\hypo{\vdash #1}}
\newcommand*{\exv}[2]{\infer{1}[\ensuremath{\mathit{ex}}]{\vdash #2}}
\newcommand*{\axIv}[1]{\infer{0}[\ensuremath{\mathit{ax1}}]{\vdash #1}}
\newcommand*{\axIIv}[1]{\infer{0}[\ensuremath{\mathit{ax2}}]{\vdash #1}}
\newcommand*{\cutv}[1]{\infer{2}[\ensuremath{\mathit{cut}}]{\vdash #1}}
\newcommand*{\onev}[1]{\infer{0}[\ensuremath{\one}]{\vdash #1}}
\newcommand*{\botv}[1]{\infer{1}[\ensuremath{\bot}]{\vdash #1}}
\newcommand*{\topv}[1]{\infer{0}[\ensuremath{\top}]{\vdash #1}}
\newcommand*{\tensorv}[1]{\infer{2}[\ensuremath{\tensor}]{\vdash #1}}
\newcommand*{\parrv}[1]{\infer{1}[\ensuremath{\parr}]{\vdash #1}}
\newcommand*{\permv}[1]{\infer{1}[\ensuremath{\sigma}]{\vdash #1}}
\newcommand*{\withv}[1]{\infer{2}[\ensuremath{\with}]{\vdash #1}}
\newcommand*{\pluslv}[1]{\infer{1}[\ensuremath{\plus_1}]{\vdash #1}}
\newcommand*{\plusrv}[1]{\infer{1}[\ensuremath{\plus_2}]{\vdash #1}}
\newcommand*{\ocv}[1]{\infer{1}[\ensuremath{\oc}]{\vdash #1}}
\newcommand*{\wkv}[1]{\infer{1}[\ensuremath{?\mathit{w}}]{\vdash #1}}
\newcommand*{\cov}[1]{\infer{1}[\ensuremath{?\mathit{c}}]{\vdash #1}}
\newcommand*{\dev}[1]{\infer{1}[\ensuremath{?\mathit{d}}]{\vdash #1}}
\newcommand*{\defv}[1]{\infer[dashed]{1}[\ensuremath{\mathit{def}}]{\vdash #1}}

\newcommand{\adaptpage}[1]{
  \setlength{\hoffset}{-0.7in}
  \setlength{\voffset}{-0.7in}
  \newsavebox{\proof}
  \sbox{\proof}{#1}
  \setlength{\pdfpageheight}{\dimexpr\ht\proof+\dp\proof+0.6in\relax}
  \setlength{\pdfpagewidth}{\dimexpr\wd\proof+0.6in\relax}
  \shipout\box\proof
}

\setlength\parindent{0.5cm}

\title{Rapport de stage}
\author{Jonas Torriero-Meskour}
\date{June 2024}

\begin{document}

\maketitle

\section{Définitions}
\subsection{Preuves}
Les séquents sont vus comme des listes. On définit alors le sous-ensemble $\mathcal{P}$ des preuves en logique linéaire que nous considérerons, par induction :
\vspace{0.5cm}

\hspace{-0.5cm}
\begin{prooftree}
  \axIv{A, {A}\orth}
\end{prooftree}
\hspace{1cm}
\begin{prooftree}
  \axIIv{{A}\orth, A}
\end{prooftree}

\vspace{0.5cm}

\hspace{-0.5cm}
\begin{prooftree}
  \hypv{\Gamma, A, B, \Delta}
  \parrv{\Gamma, A \parr B, \Delta}
\end{prooftree}
\hspace{1cm}
\begin{prooftree}
  \hypv{\Gamma, A}
  \hypv{B, \Delta}
  \tensorv{\Gamma, A \tensor B, \Delta}
\end{prooftree}
\hspace{1cm}
\begin{prooftree}
  \hypv{\Gamma}
  \permv{\Gamma_{\sigma}}
\end{prooftree}

\vspace{0.5cm}

\subsection{Représentations}
On choisit de représenter une preuve $p \in \mathcal{P}$ par un couple $(t, s) \in \mathcal{T} \times \mathcal{S}$, où $t$ est un arbre encodant le squelette de la preuve, et $s$ est le séquent prouvé.
\\

Chaque noeud de $t$ représente implicitement une règle utilisée dans $p$. Afin d'établir la correspondance, on définit une notion d'adresse, avec l'idée que chaque noeud porte une adresse qui pointe vers la sous-formule de $s$ sur laquelle la règle qu'il représente est appliquée. Le cas d'une règle axiome est particulier: un axiome est représenté par une feuille, qui a deux adresses pointant vers les deux atomes utilisés.

On pose donc l'ensemble $\mathcal{A} = \mathbb{N} \times \mathcal{L} ( \{ Left | Right\}^{*} )$, et on considère que l'adresse $(n, \rho) \in \mathcal{A}$ représente la sous-formule d'adresse $\rho$ de la $n^{eme}$ formule de $s$ (chaque formule pouvant être vue comme un arbre).
\\

On peut alors définir inductivement l'ensemble $\mathcal{T}$ des arbres, en se donnant trois constructeurs:
\begin{itemize}
  \item Un constructeur 0-aire étiquetté par deux adresses : $F: \mathcal{A} \rightarrow \mathcal{A} \rightarrow \mathcal{T}$
  \item Un constructeur unaire étiquetté par une adresse : $U: \mathcal{A} \rightarrow \mathcal{T} \rightarrow \mathcal{T}$
  \item Un constructeur binaire étiquetté par une adresse : $B: \mathcal{A} \rightarrow \mathcal{T} \rightarrow \mathcal{T} \rightarrow \mathcal{T}$
\end{itemize}



\end{document}