\documentclass[11pt,a4paper]{article}
\usepackage[T1]{fontenc}
\usepackage[utf8]{inputenc}
\usepackage[french]{babel}
\usepackage{fullpage}
\usepackage{amsmath,amsthm}
\usepackage{amssymb,cmll}
\usepackage{ebproof}
\usepackage{mathtools}
\usepackage{hyperref}
\usepackage{tikz}
\usetikzlibrary{trees}
\usepackage{bm}
\usepackage{algorithm}
\floatname{algorithm}{Algorithme}
\usepackage{algpseudocode}
\usepackage{hyperref}
\usepackage{minted}
\usepackage{authblk}
\usepackage[backend=biber,style=alphabetic,sorting=ynt]{biblatex}

\theoremstyle{plain}
\newtheorem{theorem}{Théorème}
\theoremstyle{definition}
\newtheorem{definition}{Définition}
\theoremstyle{remark}
\newtheorem{remark}{Remarque}
\newtheorem{example}{Exemple}
\newtheorem{implementation}{Implémentation}
\newtheorem{demonstrationappendix}{Démonstration annexe}

\renewcommand{\labelenumi}{(\alph{enumi})}
\renewcommand{\theenumi}{\labelenumi}

% dessins arbres : tikz
\tikzset{baseline=(current bounding box.center)}
\tikzset{level distance=1.3cm}

%imported from click and collect
\newcommand*{\orth}{^\perp}
\newcommand*{\tensor}{\otimes}

\newcommand*{\namedproofv}[2]{\hypo{#1}\infer[no rule]{1}{\vdash #2}}
\newcommand*{\hypv}[1]{\hypo{\vdash #1}}
\newcommand*{\axv}[1]{\infer{0}[\ensuremath{\mathit{ax}}]{\vdash #1}}
\newcommand*{\cutv}[1]{\infer{2}[\ensuremath{\mathit{cut}}]{\vdash #1}}
\newcommand*{\onev}[1]{\infer{0}[\ensuremath{\one}]{\vdash #1}}
\newcommand*{\botv}[1]{\infer{1}[\ensuremath{\bot}]{\vdash #1}}
\newcommand*{\topv}[1]{\infer{0}[\ensuremath{\top}]{\vdash #1}}
\newcommand*{\tensorv}[1]{\infer{2}[\ensuremath{\tensor}]{\vdash #1}}
\newcommand*{\parrv}[1]{\infer{1}[\ensuremath{\parr}]{\vdash #1}}
\newcommand*{\permv}[2]{\infer{1}[\ensuremath{\textit{ex}_{#1}}]{\vdash #2}}
\newcommand*{\withv}[1]{\infer{2}[\ensuremath{\with}]{\vdash #1}}
\newcommand*{\pluslv}[1]{\infer{1}[\ensuremath{\plus_1}]{\vdash #1}}
\newcommand*{\plusrv}[1]{\infer{1}[\ensuremath{\plus_2}]{\vdash #1}}
\newcommand*{\ocv}[1]{\infer{1}[\ensuremath{\oc}]{\vdash #1}}
\newcommand*{\wkv}[1]{\infer{1}[\ensuremath{?\mathit{w}}]{\vdash #1}}
\newcommand*{\cov}[1]{\infer{1}[\ensuremath{?\mathit{c}}]{\vdash #1}}
\newcommand*{\dev}[1]{\infer{1}[\ensuremath{?\mathit{d}}]{\vdash #1}}
\newcommand*{\defv}[1]{\infer[dashed]{1}[\ensuremath{\mathit{def}}]{\vdash #1}}
\newcommand*{\permapp}[2]{#2 #1}
\newcommand*{\someperm}{\varsigma}
\newcommand*{\someadd}{\rho}
\newcommand*{\someaddbis}{w}
\newcommand*{\someproof}{\pi}
\newcommand*{\sequent}{\Gamma}
\newcommand*{\sequentbis}{\Delta}
\newcommand*{\size}[1]{\mathopen{|}#1\mathclose{|}}

\newcommand*{\Left}{\textnormal{\texttt{L}}}
\newcommand*{\Right}{\textnormal{\texttt{R}}}

\newcommand*{\proofs}{\ensuremath{\mathcal{P}}}
\newcommand*{\sequents}{\ensuremath{\mathcal{S}}}
\newcommand*{\addresses}{\ensuremath{\mathcal{A}}}
\newcommand*{\treeaddresses}{\ensuremath{\mathcal{A'}}}
\newcommand*{\trees}{\ensuremath{\mathcal{T}}}
\newcommand*{\treespartial}{\ensuremath{\mathcal{T'}}}
\newcommand*{\representationslarge}{\ensuremath{\trees \times \sequents}}
\newcommand*{\representations}{\ensuremath{\mathcal{R}}}
\newcommand*{\representationspartiallarge}{\ensuremath{\treespartial \times \sequents}}
\newcommand*{\representationspartial}{\ensuremath{\mathcal{R'}}}
\newcommand*{\setshuffle}{\bowtie}

\newcommand*{\encode}{\ensuremath{\varphi}}
\newcommand*{\decode}{\ensuremath{\gamma}}

\newcommand*{\height}{\ensuremath{h}}
\newcommand*{\relapprox}{\ensuremath{\triangleleft}}
\newcommand*{\relapproxlarge}{\ensuremath{\relapprox^*}}
\newcommand*{\unknown}{H}

\newcommand*{\lowapprox}{\ensuremath{\Sigma_0}}
\newcommand*{\lowapproxspec}{\ensuremath{\Sigma'_0}}
\newcommand*{\highapprox}{\ensuremath{\Sigma_1}}
\newcommand*{\highapproxspec}{\ensuremath{\Sigma'_1}}

\newcommand*{\foblig}[1]{\textcolor{red}{#1}}

\newcommand*{\treesimplify}{\ensuremath{s}}

\newcommand*{\caddpartial}{(a)}
\newcommand*{\clinpartial}{(b')}
\newcommand*{\cdespartial}{(c)}

\newcommand*{\exactcond}{\bigstar_1}
\newcommand*{\exactcondbis}{\bigstar_2}

\newcommand*{\todo}{{\normalfont \textbf{TODO}} }

\bibliography{bibliographie}

\title{Conception et implémentation d'un assistant de preuve pour le fragment multiplicatif de la logique linéaire \\
}

\author{Jonas Torriero-Meskour}
\affil{ENS de Lyon, stage de L3 au LIP encadré par Olivier Laurent}

\date{4 juin 2024 -- 19 juillet 2024}

\begin{document}

%algpseudocode
\algnewcommand\algorithmicswitch{\textbf{switch}}
\algnewcommand\algorithmiccase{\textbf{case}}

\algdef{SE}[MATCHWITH]{MatchWith}{EndMatchWith}[1]{\algorithmicswitch\ #1\ \algorithmicdo}{\algorithmicend\ \algorithmicswitch}%
\algdef{SE}[CASE]{Case}{EndCase}[1]{\algorithmiccase\ #1}{\algorithmicend\ \algorithmiccase}%
\algtext*{EndMatchWith}%
\algtext*{EndCase}%

\maketitle

\tableofcontents

\section{Introduction}
La logique linéaire a été introduite en 1987 par Jean-Yves Girard~\cite{GIRARD19871}, et vise à enrichir les logiques classique et intuitionniste, en introduisant une notion de coût sur les hypothèses. En effet, dans une preuve en logique linéaire (multiplicative), chaque hypothèse doit être utilisée exactement une fois. 

De même que la logique intuitionniste se distingue de la logique classique par la notion de constructivité des preuves qu'elle apporte, la logique linéaire introduit donc une vision des hypothèses comme ressources. Elle offre en ce sens une perspective intéressante dans le cadre de l'informatique, et y a de nombreuses applications, notamment pour la programmation fonctionnelle et la théorie de la preuve~\cite{10.1093/jigpal/2.1.77} (voir également l'annexe~\ref{inst_contex}). Celles-ci impliquent souvent de construire des preuves, d'où la nécessité de rendre cette tâche simple et rapide.

Il existe plusieurs syntaxes pour construire des preuves en logique linéaire, dont le calcul des séquents. C'est la syntaxe que nous avons choisie, et pour laquelle nous proposons dans la suite un assistant de preuve. Celui-ci vise à rendre la construction des preuves plus dynamique, et partiellement automatique. 

 Notre assistant s'inscrit dans la lignée d'un outil déjà existant, Click \& Collect~\cite{callies:lirmm-03271501}, qui est l'une des premières interfaces graphiques détaillées dans la littérature permettant la construction interactive de preuves en calcul des séquents de la logique linéaire. Nous avons toutefois une visée différente, car, si Click \& Collect a un but pédagogique, puisqu'il a été conçu pour des débutants afin de leur permettre l'apprentissage de la logique linéaire, notre outil s'adresse à des utilisateurs experts, et a ainsi été conçu dans un but d'efficacité et de rapidité.

Présentons d'abord la logique linéaire plus en détail, afin de faire saisir les enjeux qui ont sous-tendu la conception de notre assistant de preuve, avant de les développer ensuite.

\subsection{Présentation de la logique linéaire}
Dans toute la suite, on ne considérera que le fragment dit \emph{multiplicatif} de la logique linéaire, appelé MLL. Comme en logique classique, les briques de base en sont des formules, qu'on définit inductivement:

\begin{definition}[Formules]
On se donne un ensemble $\mathcal{X}$ infini d'atomes. Les formules sont définies par la grammaire suivante:
\begin{equation*}
F := X \mid X\orth \mid F \parr F \mid F \tensor F
\end{equation*}
\end{definition}

\begin{remark}
    Intuitivement, un atome $X$ correspond à ce qui serait une variable en logique classique, et le constructeur $\orth$ à la négation $\neg$. La ``négation'' d'un atome s'appelle son dual. Remarquons qu'on ne peut nier que les atomes. Ce choix se justifie, si l'on pense aux lois de De Morgan en logique classique. En effet, on peut faire ``descendre'' progressivement les négations dans une formule, en appliquant successivement ces lois, jusqu'à ce que les négations ne concernent plus que les variables. La négation sur les formules au sens large n'offre donc pas d'expressivité supplémentaire, on peut ainsi s'en passer.

    Le constructeur $\parr$, appelé \emph{par}, peut être vu comme un $\lor$, et le constructeur $\tensor$, appelé \emph{tenseur}, comme un $\land$. Cela se précisera par la suite, lorsque nous parlerons des règles les concernant.
\end{remark}

Introduisons désormais les séquents, qui sont les objets pour lesquels on construit des preuves:

\begin{definition}[Séquents]
    Un séquent est une liste de formules. L'ensemble des séquents est noté $\sequents$.
\end{definition}

\begin{remark}
    Intuitivement, un séquent peut être vu comme une unique formule, qui serait, en logique classique, une disjonction des formules qu'il contient.
% Toute formule en logique classique étant équivalente à une formule en forme normale disjonctive, le fait qu'on ne puisse exprimer en logique linaire que des séquents correspondant en logique classique à des disjonctions n'est pas un problème.

    \begin{example}
    \label{example_tautology}
    Le séquent $X\orth, X$ correspond en logique classique à la formule $\neg X \lor X$, et est donc une tautologie.
    \end{example}

    Par ailleurs, on s'appuie également sur cette intuition pour justifier le fait que les séquents sont unilatéraux. En effet, en logique classique, le séquent $A \vdash B$ est équivalent au séquent $\vdash A \implies B$, qui équivaut lui-même au séquent $\vdash \neg A \lor B$. On peut ainsi l'écrire unilatéralement en logique classique, et il en va de même en logique linéaire : $A\orth, B$.
    
    L'unilatéralité des séquents ne réduit donc pas non plus l'expressivité.
\end{remark}

Avec cette idée en tête, introduisons enfin les preuves, qui définissent la notion de ``vérité'' sur les séquents:

\begin{definition}[Preuves]
\label{proofs_def}
L'ensemble des preuves \proofs{} est défini par les règles d'induction suivantes:
\begin{equation*}
\begin{prooftree}
  \axv{X\orth, X}
\end{prooftree}
\qquad\qquad
\begin{prooftree}
  \hypv{\sequent}
  \permv{\someperm}{\permapp{\someperm}{\sequent}}
\end{prooftree}
\qquad\qquad
\begin{prooftree}
  \hypv{\sequent, A, B, \sequentbis}
  \parrv{\sequent, A \parr B, \sequentbis}
\end{prooftree}
\qquad\qquad
\begin{prooftree}
  \hypv{\sequent, A}
  \hypv{B, \sequentbis}
  \tensorv{\sequent, A \tensor B, \sequentbis}
\end{prooftree}
\end{equation*}

Un séquent est prouvable s'il existe une preuve dont il est la conclusion.
\end{definition}

\begin{remark}
    La règle d'axiome $\mathit{ax}$ vient de ce qui a été dit pour l'exemple~\ref{example_tautology}: $X\orth, X$ correspond en logique classique à $\neg X \lor X$, et est donc une tautologie.

    La règle de permutation $\textit{ex}_{\someperm}$ vient d'un choix particulier que nous avons fait pour nos définitions. De manière classique, les séquents sont définis comme des \textit{multi--ensembles} de formules, mais nous avons choisi ici de les définir comme des \textit{listes}, afin de simplifier la conception de notre assistant et les démonstrations. Cette règle de permutation devient alors nécessaire, afin d'avoir un système de preuve équivalent.
    
    La règle $\parr$ est liée à une remarque faite précédemment: le par correspond à une sorte de $\lor$, et on peut donc transformer une formule $A \parr B$ en un séquent $A, B$. Toutefois, ce constructeur n'est pas inutile, car, dans des formules complexes où les opérateurs sont imbriqués, les $\parr$ ne sont pas toujours à la racine, ce qui les rend plus puissants que la simple disjonction induite par le séquent.
    
    La règle $\tensor$ vient également du lien entre $\tensor$ et $\land$, ainsi que de la notion de coût des hypothèses évoquée dans l'introduction. Intuitivement, prouver $A \tensor B$ revient à prouver $A \land B$ en logique classique, et donc à prouver $A$ et $B$ séparément. La subtilité réside dans le choix des hypothèses. Si l'on veut prouver $A \tensor B$ à partir d'un ensemble $\Sigma$ d'hypothèses, on doit en effet partitionner $\Sigma$ en deux ensembles $\sequent$ et $\sequentbis$, et prouver $A$ avec toutes les hypothèses de $\sequent$, et $B$ avec toutes celles de $\sequentbis$. On assiste ici à ce qui était évoqué dans l'introduction : chaque hypothèse de $\Sigma$ apparaît comme une ressource unique, qui doit être utilisée exactement une fois dans la preuve. 
    
    Il peut sembler, à première vue, qu'on ne puisse pas choisir comment partitionner $\Sigma$, si l'on observe uniquement la règle $\tensor$, qui n'offre aucun contrôle sur $\sequent$ et $\sequentbis$. Cependant, il faut penser cette règle en lien avec la règle $\textit{ex}_{\someperm}$, et imaginer qu'on peut avoir permuté le séquent avant de lui appliquer la règle $\tensor$, ce qui permet précisément de choisir la partition.
    
    \begin{example}[Une preuve où l'on choisit la partition]
    \begin{equation*}
        \begin{prooftree}
            \axv{{X_1}\orth, X_1}
            \axv{{X_2}\orth, X_2}
            \permv{(2, 1)}{X_2, {X_2}\orth}
            \tensorv{{X_1}\orth, X_1 \tensor X_2, {X_2}\orth}
            \permv{(3, 2, 1)}{{X_2}\orth, X_1 \tensor X_2, {X_1}\orth}
        \end{prooftree}
    \end{equation*}
    \end{example}
\end{remark}

Si le lecteur veut se familiariser plus en détail avec les règles, il peut utiliser \href{https://click-and-collect.linear-logic.org/#tutorial}{le tutoriel de Click \& Collect prévu à cet effet}, et construire par lui-même des preuves.

\subsection{Objectif de notre assistant de preuve}
Précisons d'emblée que notre objectif est d'aider l'utilisateur à construire \textit{une} preuve d'un séquent, et pas simplement de l'aider à prouver un séquent. En effet, deux preuves différentes d'un même séquent ne s'équivalent pas, et nous tenons à laisser à l'utilisateur la possibilité de construire chacune d'entre elles.

\begin{example}[Deux preuves différentes d'un même séquent]
    \begin{align*}
    \begin{multlined}
        \begin{prooftree}
                 \axv{{X_1}\orth, X_1}
                 \permv{(2, 1)}{X_1, {X_1}\orth}
                 \axv{{X_2}\orth, X_2}
              \tensorv{X_1, {X_1}\orth \tensor {X_2}\orth, X_2}
              \permv{(1, 3, 2)}{X_1, X_2, {X_1}\orth \tensor {X_2}\orth}
              \axv{{X_3}\orth, X_3}
              \permv{(2, 1)}{X_3, {X_3}\orth}
           \tensorv{X_1, X_2, ({X_1}\orth \tensor {X_2}\orth) \tensor X_3, {X_3}\orth}
        \parrv{X_1 \parr X_2, ({X_1}\orth \tensor {X_2}\orth) \tensor X_3, {X_3}\orth}
        \end{prooftree} \\ \\
        \begin{prooftree}
                 \axv{{X_1}\orth, X_1}
                 \permv{(2, 1)}{X_1, {X_1}\orth}
                 \axv{{X_2}\orth, X_2}
              \tensorv{X_1, {X_1}\orth \tensor {X_2}\orth, X_2}
              \permv{(1, 3, 2)}{X_1, X_2, {X_1}\orth \tensor {X_2}\orth}
           \parrv{X_1 \parr X_2, {X_1}\orth \tensor {X_2}\orth}
           \axv{{X_3}\orth, X_3}
           \permv{(2, 1)}{X_3, {X_3}\orth}
        \tensorv{X_1 \parr X_2, ({X_1}\orth \tensor {X_2}\orth) \tensor X_3, {X_3}\orth}
        \end{prooftree}
    \end{multlined}
    \end{align*}
    
    Dans le premier cas, on a appliqué un $\parr$ en premier, et dans le second, un $\tensor$. Cette distinction est importante et a un sens, que nous voulons donner la possibilité d'exprimer.
\end{example}

Notons qu'une solution, peu commode, pourrait être de construire automatiquement et d'afficher toutes les preuves d'un séquent donné, afin que l'utilisateur choisisse celle qui lui convienne. Ce serait néanmoins très inefficace, puisque la prouvabilité d'un séquent est un problème NP-complet pour MLL, et indécidable pour LL (qui est un fragment plus large de la logique linéaire)~\cite{Lincoln_1995}.

De ce fait, nous avons plutôt choisi de concevoir un assistant de preuve interactif, guidant l'utilisateur pas à pas, et lui faisant construire une preuve de bas en haut. L'utilisateur part ainsi d'un séquent qu'il veut prouver, et remonte petit à petit, règle après règle, jusqu'à atteindre les axiomes.

En cela, notre assistant de preuve est assez similaire à Click \& Collect. Cependant, si le lecteur l'a essayé, il s'est sans doute rendu compte de la redondance de l'information qu'il devait fournir à cet outil, à cause de la règle de permutation. Ce n'est pas dérangeant dans un but pédagogique, mais cela alourdit rapidement le travail d'un utilisateur expert. La contribution apportée par notre assistant consiste donc à réduire l'information à fournir, dans un objectif d'efficacité.

Notre assistant de preuve a été implémenté en OCaml, et peut être consulté et essayé \href{https://github.com/jonas-trms/mll_prover}{ici}. Une trace d'exécution est également donnée en annexe~\ref{exec_trace}. Nous ferons régulièrement le lien, dans la suite, entre les résultats théoriques que nous présentons, et leur implémentation dans le code.

\section{Représentation des preuves}
Si l'information requise par Click \& Collect est souvent redondante, c'est qu'il y a, en réalité, une redondance au sein même de la structure des preuves.

\begin{example}
    Pour s'en apercevoir, considérons la preuve partielle suivante (l'observation que l'on va faire serait identique si elle était complète).

    \begin{equation*}
        \begin{prooftree}
            \hypv{X_1, {X_1}\orth, X_2, {X_2}\orth}
            \parrv{X_1, {X_1}\orth \parr X_2, {X_2}\orth}
        \end{prooftree}
    \end{equation*}
    

    Si l'on regarde la définition~\ref{proofs_def}, on remarque que toute l'information de cette preuve est entièrement déterminée par la donnée du séquent de conclusion et de la règle $\parr$, et que tout le reste de l'arbre est superflu.
\end{example}

\begin{example}
    Allons plus loin, et considérons la preuve suivante:

    \begin{equation*}
        \begin{prooftree}
            \axv{{X_1}\orth, X_1}
            \axv{{X_2}\orth, X_2}
            \permv{(2, 1)}{X_2, {X_2}\orth}
            \tensorv{{X_1}\orth, X_1 \tensor X_2, {X_2}\orth}
            \permv{(3, 2, 1)}{{X_2}\orth, X_1 \tensor X_2, {X_1}\orth}
        \end{prooftree}
    \end{equation*}
    

    L'information de cette preuve est entièrement déterminée par la donnée du séquent conclusion, de la règle $\tensor$, et des deux axiomes. Les permutations peuvent s'en déduire.
\end{example}

Là est la raison de la redondance au sein de Click \& Collect, puisque cet outil demande à l'utilisateur de saisir explicitement chaque permutation, alors que leur donnée n'est en réalité jamais nécessaire.

Pour résoudre ce problème, nous proposons de travailler sur des représentations équivalentes aux arbres de preuves, qui éliminent leur redondance. Ces représentations sont inspirées des desseins utilisés par Girard dans l'introduction de sa ludique~\cite{locus}. Elles seront manipulées en arrière-plan par notre code.

\subsection{Représentations}
Nous choisissons de représenter une preuve $\someproof \in \mathcal{P}$ par un couple $(t, \sequent) \in \representationslarge$, où $t$ est un arbre encodant le squelette de la preuve, et $\sequent$ est le séquent prouvé par $\someproof$. Définissons cela plus en détail, en commençant par nous donner un ensemble d'adresses:

\begin{definition}[Adresses]
On pose:
% \begin{equation*}
\quad $\mathcal{A} = \mathbb{N} \times \{ \Left, \Right\}^{*}$
%\end{equation*}

Dans un séquent $\sequent$, l'adresse $(n, \rho) \in \mathcal{A}$ (notée souvent $n \rho$) servira ainsi à représenter la sous-formule d'adresse $\rho$ de la $n$\ieme{} formule de $\sequent$, chaque formule pouvant être vue comme un arbre binaire.
\end{definition}

\begin{example}
Considérons le séquent suivant:
%\begin{equation*}
\quad $X_1\orth, X_1 \tensor (X_2 \tensor X_3), X_3\orth, X_2\orth$
%\end{equation*}

$(1, \epsilon)$ représente la sous-formule $X_1\orth$, et $(2, \Right \cdot \Left)$ représente la sous-formule $X_2$. $(2, \Right)$ représente la sous-formule $X_2 \tensor X_3$.
\end{example}

Nous pouvons maintenant définir les arbres, qui serviront à représenter les squelettes des preuves:

\begin{definition}[Arbres]
\label{def_trees}
L'ensemble \trees{} des arbres est défini inductivement, par trois constructeurs:
\begin{itemize}
  \item un constructeur $0$-aire étiqueté par deux adresses: $F: \mathcal{A} \rightarrow \mathcal{A} \rightarrow \trees$
  \item un constructeur unaire étiqueté par une adresse: $U: \mathcal{A} \rightarrow \trees \rightarrow \trees$
  \item un constructeur binaire étiqueté par une adresse: $B: \mathcal{A} \rightarrow \trees \rightarrow \trees \rightarrow \trees$
\end{itemize}
\end{definition}

Dans une représentation, chaque n\oe ud de l'arbre symbolisera ainsi une règle de la preuve. En effet, chaque n\oe ud portera une adresse qui renverra à une sous-formule de $\sequent$, et on remarque, en s'attardant sur la définition~\ref{proofs_def}, qu'une sous-formule de $\sequent$ correspond exactement à une application de règle.

Le cas de l'axiome est particulier: un axiome sera représenté par une feuille ayant deux adresses, qui renverront aux deux atomes duaux concernés. À nouveau, en regardant la définition~\ref{proofs_def}, on voit que cette information est équivalente à la donnée de l'axiome.

\begin{example}
\label{rep_example}
Une preuve et sa représentation (la racine de la preuve est positionnée en bas, celle de l'arbre de la représentation est positionnée en haut):
\begin{equation*}
\begin{prooftree}
    \axv{{X_1}\orth, X_1}
    \axv{{X_2}\orth, X_2}
    \permv{(2, 1)}{X_2, {X_2}\orth}
    \tensorv{{X_1}\orth, X_1 \tensor X_2, {X_2}\orth}
    \parrv{{X_1}\orth \parr (X_1 \tensor X_2), {X_2}\orth}
\end{prooftree}
\quad\leadsto\quad
\begin{tikzpicture}%
    [level 2/.style={sibling distance=3.5cm}]
    \node {$U_{1 \epsilon}$}
    child {node {$B_{1 \Right}$}
        child {node {$F_{1 \Left, \; 1 \Right \cdot \Left}$}}
        child {node {$F_{1 \Right \cdot \Right, \; 2 \epsilon}$}}
    };
\end{tikzpicture}
\vdash {X_1}\orth \parr (X_1 \tensor X_2), {X_2}\orth
\end{equation*}
\end{example}

\begin{implementation}
    Ces structures sont implémentées dans notre code par:
    
    \label{implement_trees}
    \begin{minted}{ocaml}
        type formula, type sequent, type proof
        type tree, type representation = tree * sequent
    \end{minted}

    \begin{remark}
        Les arbres que nous implémentons ne sont pas exactement ceux dont nous venons de parler, mais une version qui leur est plus générale, et dont nous parlerons quand nous nous intéresserons à la manipulation interactive des représentations, dans la section~\ref{partial_rep}.
    \end{remark}

\end{implementation}

\subsection{Traduction}
Prouvons maintenant que nos représentations sont équivalentes aux arbres de preuve, en construisant des fonctions d'encodage et de décodage des preuves. On devra, au passage, ajouter des contraintes supplémentaires sur les représentations, afin de ne conserver que celles qui sont correctes.

\subsubsection{Encodage}

On identifie dans toute la suite une fonction de $\addresses$ dans $\addresses$ et son extension de $\trees$ dans $\trees$, qui remplace chaque adresse de l'arbre par son image. Donnons directement la fonction d'encodage:

\begin{definition}[Fonction d'encodage]
  On définit d'abord une fonction $\encode' : \proofs \rightarrow \trees$, par induction sur $\proofs$:
    \begin{description}
    \item[Axiome:]
    $\encode' \left(
    \begin{prooftree}
        \axv{{X}\orth, X}
    \end{prooftree}
    \right) = F(1 \epsilon, \; 2 \epsilon)$

    \item[Échange:]
    $\encode' \left(
    \begin{prooftree}
      \namedproofv{\pi}{\sequent}
      \permv{\someperm}{\permapp{\someperm}{\sequent}}
    \end{prooftree}
    \right) = \someperm' \left( \encode ' \left(
           \begin{prooftree}
             \namedproofv{\pi}{\sequent}
           \end{prooftree} \right) \right)$
           
    où $\someperm'$ est la fonction qui applique $\someperm$ à l'arbre (on étend d'abord $\someperm$ aux adresses, en considérant que $\someperm$ s'applique à l'entier dans le couple), puis trie les deux adresses de chaque feuille par ordre lexicographique.

    \item[Par:]
    $\encode' \left(
    \begin{prooftree}
      \namedproofv{\pi}{\sequent, A, B, \sequentbis}
      \parrv{\sequent, A \parr B, \sequentbis}
    \end{prooftree}
    \right) = \begin{tikzpicture}%
    [level 2/.style={sibling distance=3.5cm}]
    \node {$U_{n \epsilon}$}
        child {node {$\psi_\parr \left( \encode' \left(
           \begin{prooftree}
             \namedproofv{\pi}{\sequent, A, B, \sequentbis}
           \end{prooftree} \right) \right)$}
    };
    \end{tikzpicture}$
    
    où $n = \size{\sequent} + 1$
    
    $\psi_\parr : \addresses \rightarrow \addresses =
    \begin{cases*}
        (i, \someadd) \mapsto (i, \someadd) & si $i < n$ \\
        (n, \someadd) \mapsto (n, \Left \cdot \someadd)\\
        (n+1, \someadd) \mapsto (n, \Right \cdot \someadd)\\
        (i, \someadd) \mapsto (i-1, \someadd) & si $i \geq n+2$
    \end{cases*}$

    \item[Tenseur:]
    $\encode' \left(
    \begin{prooftree}
      \namedproofv{\pi_1}{\sequent, A}
      \namedproofv{\pi_2}{B, \sequentbis}
      \tensorv{\sequent, A \tensor B, \sequentbis}
    \end{prooftree}
    \right) = \begin{tikzpicture}%
    [level 1/.style={sibling distance=3.5cm}]
    \node {$B_{n \epsilon}$}
        child {node {$\psi_{\tensor\Left} \left( \encode' \left(
                \begin{prooftree}
                  \namedproofv{\pi_1}{\sequent, A}
                \end{prooftree}
              \right) \right)$}}
        child {node {$\psi_{\tensor\Right} \left( \encode' \left(
                \begin{prooftree}
                  \namedproofv{\pi_2}{B, \sequentbis}
                \end{prooftree}
              \right) \right)$}
    };
    \end{tikzpicture}$
    
    où $n = \size{\sequent} + 1$
    
    $\psi_{\tensor\Left} : \addresses \rightarrow \addresses =
    \begin{cases*}
        (n, \someadd) \mapsto (n, \Left \cdot \someadd)\\
        (i, \someadd) \mapsto (i, \someadd) & si $i \neq n$
    \end{cases*}$
    
    $\psi_{\tensor\Right} : \addresses \rightarrow \addresses =
    \begin{cases*}
        (1, \someadd) \mapsto (n, \Right \cdot \someadd)\\
        (i, \someadd) \mapsto (i + n - 1, \someadd) & si $i \geq 2$
    \end{cases*}$
  \end{description}

    Puis on définit la fonction d'encodage $\encode : \proofs \rightarrow \representationslarge$:
    \begin{equation*}
    \forall \someproof \in \proofs, \encode \left( \someproof \right) = \left( \encode' \left( \someproof \right), \; \sequent \right), \; \text{noté $\encode' \left( \someproof \right) \vdash \sequent$, où $\sequent$ est le séquent prouvé par $\someproof$.}
    \end{equation*}
\end{definition}

\begin{remark}
    On préserve l'ordre lexicographique sur les feuilles, car on souhaite étiqueter celles-ci par des paires d'adresses et non des couples.
% La règle d'axiome étant unique, elle ne demande pas d'ordre sur les atomes, ce qui fait de leur paire une information suffisante.

    Pour des questions d'implémentation, on réalise en pratique les paires par des couples ordonnés, d'où l'ordre lexicographique.
\end{remark}

\begin{remark}
    On vérifie au passage que l'encodage de l'exemple~\ref{rep_example} par notre fonction correspond bien à la représentation qu'on en avait donnée.
\end{remark}

\begin{implementation}
    Cette fonction d'encodage est implémentée dans notre code par:
    \begin{minted}{ocaml}
    let encode proof = ...
    \end{minted}
\end{implementation}

Si l'on regarde désormais l'ensemble des représentations $\representationslarge$ de plus près, on remarque un problème: il est en l'état trop souple pour être équivalent aux preuves, ce qui laisse à penser que l'ensemble d'arrivée de $\encode$ est en réalité plus petit.

\begin{example}
    Pour le remarquer, considérons la représentation suivante:
    
    \begin{equation*}
        \begin{tikzpicture}%
        [level 2/.style={sibling distance=3.5cm}]
        \node {$U_{42 \epsilon }$}
        child {node {$B_{13 \Right \cdot \Right \cdot \Right}$}
            child {node {$F_{1 \epsilon, \; 1 \epsilon}$}}
            child {node {$F_{1 \epsilon, \; 1 \epsilon}$}}
        };
        \end{tikzpicture}
        \vdash {X_1}\orth \parr (X_1 \tensor X_2), {X_2}\orth
    \end{equation*}

    Elle appartient bien à notre ensemble $\representationslarge$, puisque celui-ci n'impose aucune contrainte de lien entre l'arbre et le séquent, mais on se rend vite compte qu'elle n'a aucun sens, ses adresses étant absurdes. Elle ne peut pas être l'image d'une preuve.
\end{example}

On impose donc des contraintes sur nos représentations, afin de restreindre leur ensemble et de ne conserver que celles qui ``ont du sens''. Cette notion ``d'avoir du sens'', qu'on définit par nos contraintes, va correspondre au fait d'être l'encodage d'une preuve (théorèmes~\ref{encode_img} et ~\ref{rep_inversion}).

\begin{definition}[Représentations correctes]
    \label{def_rep}
    L'ensemble \representations{} des représentations correctes est l'ensemble des éléments $(t, \sequent)$ de $\representationslarge$ vérifiant les trois conditions suivantes:
    
    \begin{enumerate}
    \item\label{cadd} \textbf{Bon adressage:} Chaque adresse apparaissant dans une feuille ou un n\oe ud de $t$ est celle d'un atome ou d'un opérateur de $\sequent$. De plus, l'arité d'un n\oe ud est égale à celle de la règle de l'opérateur qu'il adresse, et chaque feuille adresse deux atomes duaux. Enfin, les deux adresses d'une feuille sont triées (par ordre lexicographique, en considérant que $\Left < \Right)$.
    \item\label{clin} \textbf{Linéarité:} Chaque sous-formule de $\sequent$ apparaît exactement une fois dans les adresses de $t$.
    \item\label{cdes} \textbf{Descendance:} Une adresse de $t$ de la forme $(n, \someadd \cdot \alpha)$ descend nécessairement d'un n\oe ud adressant $(n, \someadd)$ (possiblement à distance plus grande que $1$). De plus, pour tout n\oe ud de la forme $B((n, \someadd), \; g, \; d)$, $\; (n, \someadd \cdot \Right)$ ne peut apparaître dans $g$, et $(n, \someadd \cdot \Left)$ ne peut apparaître dans $d$.
    \end{enumerate}
\end{definition}

Montrons désormais ce qui vient d'être évoqué : le fait d'être une représentation correcte correspond exactement au fait d'être l'encodage d'une preuve. Énonçons un premier théorème, justifiant le premier sens:

\begin{theorem}[Image de l'encodage]
    \label{encode_img}
    \begin{equation*}
    \forall \someproof \in \proofs, \encode \left( \someproof \right) \in \representations
    \end{equation*}
%    Autrement dit, l'encodage d'une preuve est bien une représentation correcte.
\end{theorem}

\begin{proof}
    Par induction sur $\someproof \in \proofs$. La preuve est relativement directe, puisqu'il suffit de vérifier les trois conditions pour chaque cas. Elle se trouve en démonstration annexe~\ref{proof_img_encode}.
\end{proof}

\subsubsection{Décodage}
Il reste à montrer l'autre sens, à savoir qu'une représentation correcte est bien l'encodage d'une preuve. On énonce pour cela un second théorème, le justifiant:

\begin{theorem}[Inversion des représentations correctes]
\label{rep_inversion}
\begin{equation*}
\forall (t, \sequent) \in \representations, \; \exists \someproof \in \proofs, \; \encode \left( \someproof \right) = (t, \sequent)
\end{equation*}
% Autrement dit, une représentation correcte est bien l'encodage d'une preuve.
\end{theorem}

\begin{proof}
    Par récurrence sur la hauteur $\height$ de $t \in \trees$. On construit un antécédent dans $\proofs$. La preuve complète est en démonstration annexe~\ref{proof_rep_inversion}.
\end{proof}

\begin{remark}
    La preuve que l'on donne est constructive, et c'est important. En effet, la représentation devra être décodée une fois les manipulations achevées, afin que la preuve soit affichée.
\end{remark}

\begin{implementation}
    Une fonction de décodage est donc implémentée dans notre code, par:
    \begin{minted}{ocaml}
    let decode proof = ...
    \end{minted}
\end{implementation}

\begin{remark}
    \label{remark_kernel}
    Nous venons de montrer que l'espace d'arrivée de la fonction d'encodage est $\representations$, et qu'elle est surjective. Une question reste de savoir si elle injective, puisque nous aimerions avoir un système de représentation équivalent aux preuves.

    Ce n'est en fait pas le cas, mais cela ne pose pas problème. En effet, notre fonction d'encodage est injective \textit{à permutations intermédiaires près}. Cela signifie que deux preuves ont la même image ssi l'une peut être obtenue à partir de l'autre en lui ajoutant des règles de permutation intermédiaires, c'est-à-dire des règles qui ne modifient ni le séquent conclusion ni les axiomes.

    Cette injectivité particulière vient du choix que nous avons fait pour nos définitions, mentionné dans la définition~\ref{proofs_def}: en réalité, les séquents sont des \textit{multi--ensembles} de formules, mais nous les représentons par des \textit{listes}. Il est donc cohérent que nous identifiions les preuves identiques à permutations intermédiaires près, car elles représentent toutes, en réalité, la même preuve. On pourrait même identifier les preuves ayant des conclusions identiques à permutation près, mais nous ne le faisons pas pour des raisons pratiques.

    \begin{example}Deux preuves partielles identiques à permutations intermédiaires près:
        \begin{equation*}
            \begin{prooftree}
               \hypv{{X_1}\orth, X_2, X_3, {X_4}\orth}
               \parrv{{X_1}\orth, X_2 \parr X_3, {X_4}\orth}
            \end{prooftree}
\qquad\qquad
             \begin{prooftree}
               \hypv{{X_1}\orth, X_2, X_3, {X_4}\orth}
               \permv{(3, 1, 2, 4)}{X_2, X_3, {X_1}\orth, {X_4}\orth}
               \parrv{X_2 \parr X_3, {X_1}\orth, {X_4}\orth}
               \permv{(2, 1, 3)}{{X_1}\orth, X_2 \parr X_3, {X_4}\orth}
            \end{prooftree}
        \end{equation*}
    \end{example}
\end{remark}


\section{Manipulation interactive des représentations}
Nous venons de donner et de justifier une représentation équivalente aux preuves, mais, lors de l'interaction, l'utilisateur travaille essentiellement sur des preuves partielles. Il nous faut donc généraliser nos représentations à des représentations partielles, qui pourront être manipulées.

\subsection{Représentations partielles}
\label{partial_rep}

Celles-ci seront identiques aux représentations complètes, à la différence que leurs arbres contiendront éventuellement des trous, aux endroits où on ne saura pas encore quoi mettre. On étend en ce sens la définition~\ref{def_trees}:

\begin{definition}[Arbres partiels]
    L'ensemble \treespartial{} des arbres partiels est défini inductivement, par quatre constructeurs:
    \begin{itemize}
      \item un constructeur $0$-aire, le \emph{trou}: $\unknown: \treespartial$
      \item un constructeur $0$-aire étiqueté par deux adresses: $F: \mathcal{A} \rightarrow \mathcal{A} \rightarrow \treespartial$
      \item un constructeur unaire étiqueté par une adresse: $U: \mathcal{A} \rightarrow \treespartial \rightarrow \treespartial$
      \item un constructeur binaire étiqueté par une adresse: $B: \mathcal{A} \rightarrow \treespartial \rightarrow \treespartial \rightarrow \treespartial$
    \end{itemize}
\end{definition}

\begin{implementation}
    Voir le point d'implémentation~\ref{implement_trees}. Les arbres plus généraux dont on parlait sont ceux que nous venons de définir.
\end{implementation}

\begin{remark}
    Les arbres complets \trees{} sont en particulier des arbres partiels appartenant à \treespartial{}.
\end{remark}

\begin{remark}
    Le constructeur $\unknown$ correspond à ce qui a été expliqué: il s'agit d'une partie de l'arbre qu'on ne connaît pas encore. Au cours de l'interaction, on les éliminera peu à peu, au fur et à mesure qu'on gagnera de l'information.
\end{remark}

Afin de formaliser cette notion de ``gagner de l'information'', dont on aura besoin plus tard pour justifier une correspondance entre les représentations et les preuves partielles, on définit une relation d'approximation sur \treespartial{}:

\begin{definition}[Relation d'approximation]
    Soient $t', t \in \treespartial$. On dit que $t'$ est une approximation immédiate de $t$, qu'on note $t' \relapprox t$, ssi $t$ s'obtient en remplaçant un trou $\unknown$ quelconque de $t'$ par un constructeur de la forme $F(a_1, a_2)$, $U(a_1, \unknown)$ ou $B(a_1, \unknown, \unknown)$, avec $a_1, a_2 \in \addresses$.
\end{definition}

\begin{remark}
    On étend au passage cette relation à sa clôture réflexive et transitive, notée $\relapproxlarge$.
\end{remark}

\begin{example}
Une suite d'approximations immédiates:
% (pour la lisibilité, les séquents ne sont pas représentés):
    \begin{equation*}
    \begin{tikzpicture}%
        [level 2/.style={sibling distance=3.5cm}]
        \node {$U_{1 \epsilon}$}
        child {node {$\unknown$}};
    \end{tikzpicture}
    \relapprox
    \begin{tikzpicture}%
        [level 2/.style={sibling distance=3.5cm}]
        \node {$U_{1 \epsilon}$}
        child {node {$B_{1 \Right}$}
            child {node {$\unknown$}}
            child {node {$\unknown$}}
        };
    \end{tikzpicture}
    \relapprox
    \begin{tikzpicture}%
        [level 2/.style={sibling distance=3.5cm}]
        \node {$U_{1 \epsilon}$}
        child {node {$B_{1 \Right}$}
            child {node {$\unknown$}}
            child {node {$F_{1 \Right \cdot \Right, \; 2 \epsilon}$}}
        };
    \end{tikzpicture}
    \end{equation*}
\end{example}

À nouveau, on ne souhaite s'intéresser qu'à des représentations partielles qui ``ont du sens''. Cette fois, ``avoir du sens'' signifiera ne pas être une représentation partielle dont on peut déjà savoir qu'elle est incorrecte. On étend donc la définition~\ref{def_rep}, en imposant à nouveau des contraintes:

\begin{definition}[Représentations partielles correctes]
    \label{def_rep_partial}
    L'ensemble \representationspartial{} des représentations partielles correctes est l'ensemble des éléments $(t, \sequent)$ de $\representationspartiallarge$ vérifiant les trois conditions suivantes:
    
    \begin{enumerate}
    \item[\caddpartial] \textbf{Bon adressage:} Même condition que \ref{cadd} de la définition~\ref{def_rep}, en considérant que les feuilles $\unknown$ ne portent aucune adresse.
    \item[\clinpartial] \textbf{Sous-linéarité:} Chaque sous-formule de $\sequent$ apparaît \emph{au plus} une fois dans les adresses de $t$.
    \item[\cdespartial] \textbf{Descendance:} Même condition que \ref{cdes} de la définition~\ref{def_rep}.
    \end{enumerate}
\end{definition}

\begin{remark}
    Ces contraintes sont bien des extensions directes des contraintes pour les représentations complètes correctes.
\end{remark}

Les représentations partielles correctes ont bien le sens dont on vient de parler, comme le justifie la contraposée du théorème suivant:

\begin{theorem}
    \label{approxofcorrect}
    Soit $(t, \sequent) \in \representationslarge$ tel que $(t, \sequent) \in \representations$, et soit $t' \in \treespartial$ tel que $t' \relapproxlarge t$. Alors $(t', \sequent) \in \representationspartial$.
    
    Autrement dit, une approximation d'une représentation correcte est une représentation partielle correcte.
\end{theorem}

\begin{remark}
    Par contraposée, une représentation partielle incorrecte n'est donc pas l'approximation d'une représentation complète correcte.
\end{remark}

\begin{proof}
    Les trois conditions de la définition~\ref{def_rep} sont vérifiées pour $(t, \sequent)$, et en particulier les conditions de la définition~\ref{def_rep_partial}, qui en sont une version plus faible. 
    
    $t'$ s'obtient en remplaçant un nombre fini de noeuds de $t$ par des feuilles $\unknown$, en partant des feuilles et en percolant vers le haut. Cette opération préserve les trois conditions de la définition~\ref{def_rep_partial}:

    \begin{enumerate}
        \item[\caddpartial] On a seulement retiré des adresses, les adresses qui restent dans $t'$ sont donc correctes car elles l'étaient déjà dans $t$.

        \item[\clinpartial] De même, on n'a fait que retirer des adresses, et on doit vérifier que les adresses qui restent dans $t'$ sont distinctes. C'est vrai, car les adresses de $t$ étaient déjà distinctes.

        \item[\cdespartial] On retire les adresses en percolant vers le haut depuis les feuilles. On ne se trouve donc jamais dans le cas où l'on retire une adresse ayant des descendants, ce qui préserve la condition de descendance.
\qedhere
    \end{enumerate}
\end{proof}

\begin{remark}
    En particulier, pour tout séquent $\sequent$, $(\unknown, \; \sequent) \in \representationspartial$. 
    
    Cette observation est cohérente, puisqu'un arbre réduit à un trou est un arbre ne contenant aucune information. Il s'agit en fait d'une représentation partielle correcte minimale.
\end{remark}

\subsection{Construction interactive d'une preuve}

Les représentions partielles étant introduites, nous avons désormais tous les éléments nécessaires au fonctionnement de notre assistant. Nous pouvons ainsi donner sa méthode, qui guide l'utilisateur interactivement et pas à pas.

La construction d'une preuve se fait de la manière suivante : l'utilisateur donne en entrée un séquent, qu'il veut prouver, et part de la représentation partielle correcte minimale $(\unknown, \; \sequent)$. 

Étape par étape, il remplace ensuite les trous de l'arbre, dans l'ordre de son choix, par des nouveaux n\oe{}uds, et construit ainsi une suite d'approximations immédiates. Cette suite aboutit, si le séquent est prouvable, et que l'utilisateur a fait les bons choix, à une représentation totale correcte. L'assistant la décode, et affiche alors la preuve construite (voir trace en annexe~\ref{exec_trace}).

\begin{example}
    Construction d'une preuve pour le séquent ${X_1}\orth \parr (X_1 \tensor X_2), {X_2}\orth$ (qui ne sera pas représenté pour la lisibilité):

    L'utilisateur part de l'arbre vide, et construit, guidé par notre assistant, une suite d'approximations:

    \begin{align*}
    \begin{multlined}
        \unknown 
        \relapprox
        \begin{tikzpicture}%
            [level 2/.style={sibling distance=3.5cm}]
            \node {$U_{1 \epsilon}$}
            child {node {$\unknown$}
            };
        \end{tikzpicture}
        \relapprox
        \begin{tikzpicture}%
            [level 2/.style={sibling distance=3.5cm}]
            \node {$U_{1 \epsilon}$}
            child {node {$B_{1 \Right}$}
                child {node {$\unknown$}}
                child {node {$\unknown$}}
            };
        \end{tikzpicture}
        \relapprox
        \begin{tikzpicture}%
            [level 2/.style={sibling distance=3.5cm}]
            \node {$U_{1 \epsilon}$}
            child {node {$B_{1 \Right}$}
                child {node {$F_{1 \Left, \; 1 \Right \cdot \Left}$}}
                child {node {$\unknown$}}
            };
        \end{tikzpicture}
        \relapprox
        \begin{tikzpicture}%
            [level 2/.style={sibling distance=3.5cm}]
            \node {$U_{1 \epsilon}$}
            child {node {$B_{1 \Right}$}
                child {node {$F_{1 \Left, \; 1 \Right \cdot \Left}$}}
                child {node {$F_{1 \Right \cdot \Right, \; 2 \epsilon}$}}
            };
        \end{tikzpicture}
    \end{multlined}
    \end{align*}

    Il aboutit à la représentation correcte complète suivante:
    \begin{equation*}
        \begin{tikzpicture}%
            [level 2/.style={sibling distance=3.5cm}]
            \node {$U_{1 \epsilon}$}
            child {node {$B_{1 \Right}$}
                child {node {$F_{1 \Left, \; 1 \Right \cdot \Left}$}}
                child {node {$F_{1 \Right \cdot \Right, \; 2 \epsilon}$}}
            };
        \end{tikzpicture}
        \vdash {X_1}\orth \parr (X_1 \tensor X_2), {X_2}\orth
    \end{equation*}
    
    On la décode, et on la lui affiche finalement en tant que preuve:
    \begin{equation*}
        \begin{prooftree}
            \axv{{X_1}\orth, X_1}
            \axv{{X_2}\orth, X_2}
            \permv{(2, 1)}{X_2, {X_2}\orth}
            \tensorv{{X_1}\orth, X_1 \tensor X_2, {X_2}\orth}
            \parrv{{X_1}\orth \parr (X_1 \tensor X_2), {X_2}\orth}
        \end{prooftree}
    \end{equation*}
\end{example}

\begin{implementation}
    Cette boucle d'interaction est implémentée dans notre code par:

    \begin{minted}{ocaml}
        let prove_sequent s = ...
    \end{minted}
\end{implementation}

\subsection{Approximations d'adresses}

Une difficulté que l'on remarque est que l'utilisateur doit faire les bons choix. Notre objectif est de le guider au mieux dans cette démarche, en lui indiquant, pour chaque trou, un ensemble de n\oe{}uds parmi lesquels il peut piocher, et en surlignant dans cet ensemble certains n\oe{}uds obligatoires. Nous calculons ainsi des approximations basse et haute d'adresses, qui permettront de justifier une correction et une complétude de notre assistant.

\begin{remark}
    Nos ensembles ne contiennent en pratique que des adresses, puisque, pour un séquent et une adresse donnés, un unique constructeur que nous pouvons retrouver leur est associé. 
\end{remark}

\begin{implementation}
    En pratique, au cours des interactions, les représentations partielles sont affichées dans le terminal comme du pseudo-code \LaTeX. L'utilisateur choisit, en donnant un indice numérique, le trou sur lequel il veut travailler. Un séquent est ensuite affiché, contenant les sous-formules liées aux adresses qu'il peut choisir. L'utilisateur fait son choix en donnant un nouvel indice numérique, indiquant la sous-formule qu'il pioche dans le séquent.

    Pour s'en faire une idée plus concrète, on peut consulter la trace d'exécution donnée en annexe~\ref{exec_trace}.
\end{implementation}

\subsubsection{Calcul des approximations}

Expliquons comment nous calculons nos approximations, avant de les spécifier. Introduisons d'abord une notion utile:

\begin{definition}[Adresse dans un arbre]
    On définit une notion d'adresse dans les arbres, qui servira à adresser les n\oe uds. On pose ainsi l'ensemble $\mathcal{A'} = \{ \Left, \Right\}^{*}$, et on utilise une convention pour le cas du n\oe ud unaire: son descendant est considéré comme son fils gauche.
\end{definition}

\begin{example} Considérons l'arbre suivant:
    \begin{equation*}
        \begin{tikzpicture}%
        [level 2/.style={sibling distance=5cm}
        level 3/.style={sibling distance=10cm}]
        \node {$U_{1 \epsilon}$}
        child {node {$B_{1 \Right}$}
            child {node {$F_{1 \Left, \; 1 \Right \cdot \Left}$}}
            child {
                node {$B_{2 \epsilon}$}
                child {node {$F_{1 \Right \cdot \Right, \; 2 \epsilon}$}}
                child {node {$F_{3 \epsilon, \; 4 \epsilon}$}}
        }};
        \end{tikzpicture}
    \end{equation*}

    L'adresse $\epsilon$ renvoie à la racine $U_{1 \epsilon}$, et l'adresse $\Left \cdot \Right$ au n\oe ud $B_{2 \epsilon}$. L'adresse $\Left \cdot \Right \cdot \Right$ renvoie à la feuille $F_{3 \epsilon, \; 4 \epsilon}$.
\end{example}

Donnons maintenant l'algorithme d'approximation: algorithme~\ref{algo_approx}.

\begin{algorithm}
\caption{Calcul des approximations basse et haute d'adresses}\label{algo_approx}
\begin{algorithmic}
    \Function{approx}{$t, \; a', \; \highapprox, \; \lowapprox$}
        \MatchWith{$t, \; a'$}
            \Case{$H, \; \epsilon$}
                \State \Return $\highapprox, \lowapprox$
            \EndCase
    
            \Case{$U((m, \someadd), t'), \; \Left \cdot \someaddbis$}
                \State $\highapprox \gets ({\highapprox}\setminus \{ (m, \someadd) \}) \cup \left\{ (m, \someadd \cdot \Left), (m, \someadd \cdot \Right)\right\}$
                
                \State $\lowapprox \gets ({\lowapprox}\setminus \{ (m, \someadd) \}) \cup \left\{ (m, \someadd \cdot \Left), (m, \someadd \cdot \Right)\right\}$
                
                \State \Return $\text{APPROX}(t', \; \someaddbis, \; \highapprox, \; \lowapprox)$
            \EndCase
    
            \Case{$B((m, \someadd), t_1, t_2), \; \Left \cdot \someaddbis$}
                \State $\mathcal{M} \gets \left\{\text{$a$, où $a$ est une adresse minimale apparaissant dans $t_2$}\right\}$
                
                \State $\highapprox \gets ({\highapprox}\setminus (\mathcal{M} \cup \{ (m, \someadd) \})) \cup \left\{ (m, \someadd \cdot \Left) \right\}$
    
                \If{$t_2 \in \trees$}
                    \State $\lowapprox \gets ({\lowapprox}\setminus (\mathcal{M} \cup \{ (m, \someadd) \})) \cup \left\{ (m, \someadd \cdot \Left) \right\}$
                \Else
                    \State $\lowapprox \gets \left\{ (m, \someadd \cdot \Left) \right\}$
                \EndIf
                
                \State \Return $\text{APPROX}(t', \; \someaddbis, \; \highapprox, \; \lowapprox)$
            \EndCase
    
            \Case{$B((m, \someadd), t_1, t_2), \; \Right \cdot \someaddbis$}
                \State Cas symétrique.
            \EndCase
        \EndMatchWith
    \EndFunction
\end{algorithmic}
\end{algorithm}

\begin{remark}
    \label{min_address}
    Une adresse minimale dans $t_2$ est une adresse $(n, \someadd)$ n'ayant pas d'ancêtre de la forme $(n, \someadd')$, où $\someadd = \someadd' \cdot \alpha$.
\end{remark}

\begin{remark}
    $\highapprox$ est l'approximation haute, et correspond aux adresses qu'on peut choisir pour remplacer le trou. $\lowapprox$ est l'approximation basse, et correspond à des adresses obligatoires dans l'approximation haute.
On a $\lowapprox\subseteq\highapprox$.
\end{remark}

\begin{remark}
    \label{algo_expl}
    Notre algorithme est récursif. L'idée est que l'on descend le long de la branche racine--trou, en faisant à chaque fois comme si le n\oe ud sur lequel on se trouve était un trou. On calcule pour celui-ci les deux approximations d'adresses, en mettant à jour celles qu'on avait calculées pour son père.
\end{remark}

\begin{remark}
    Dans les cas où l'on appelle APPROX sur $t \in \treespartial$ et $a \in \treeaddresses$, tels que $a$ adresse un trou dans $t$, l'algorithme termine. Ce cas est celui qui nous intéresse, puisqu'on cherchera toujours à calculer des approximations d'adresses pour un trou.
    
    Dans ce cas, on vérifie qu'on rentre à chaque appel dans une des clauses du switch. De plus, la longueur de l'adresse $a'$ décroît strictement au fil des appels, et est minorée par 0 ($a' = \epsilon$ est un cas de base).
\end{remark}

\begin{implementation}
    Cet algorithme est implémenté dans notre code par:
    \begin{minted}{ocaml}
    let approx (t, s) a = ...
    \end{minted}
\end{implementation}

\subsubsection{Spécification des approximations}

Comme notre algorithme termine dans ses cas d'application, nous pouvons le spécifier, en commençant par l'approximation haute:

\begin{definition}[Spécification de l'approximation haute]
    \label{lowapprox_branch}
    Soient $(t, \; \sequent) \in \representationspartial$ et $a' \in \treeaddresses$, tels que $t$ a un trou adressé par $a'$.

    On pose $\highapproxspec \left( t, \; \sequent, \; a' \right)$ l'ensemble des adresses n'apparaissant pas dans $t$, qui sont des adresses racines correctes, ou des adresses correctes descendant immédiatement (en respectant les conditions gauche et droite de descendance) d'une adresse présente sur la branche racine--trou de $t$.
\end{definition}

\begin{remark}
     On ne peut pas avoir $t \in \trees$.
\end{remark}

\begin{remark}
    Une adresse racine est une adresse de la forme $(n, \; \epsilon)$.
    
    Une adresse correcte est une adresse pointant vers une sous-formule de $\sequent$.
\end{remark}

\begin{example}Considérons la représentation suivante:
    \begin{equation*}
        \begin{tikzpicture}%
            [level 2/.style={sibling distance=3.5cm}]
            \node {$U_{1 \epsilon}$}
            child {node {$B_{1 \Right}$}
                child {node {$F_{1 \Left, \; 1 \Right \cdot \Left}$}}
                child {node {$H$}}
            };
        \end{tikzpicture}
        \vdash {X_1}\orth \parr (X_1 \tensor X_2), {X_2}\orth
    \end{equation*}

    L'ensemble $\highapproxspec$ contient les adresses $1 \Right \cdot \Right$ et $2 \epsilon$.
\end{example}

\begin{remark}
     Cet ensemble est celui des adresses possibles pour le trou. On montrera, en justifiant la correction de l'assistant, que les utiliser pour substituer le trou conserve la correction partielle de la représentation (voir section~\ref{correction_section}).
\end{remark}

\begin{theorem}[Correction de l'approximation haute]
    \label{highapprox_correction}
    
    Soient $(t, \; \sequent) \in \representationspartial$, et $a' \in \treeaddresses$, tels que $a'$ adresse un trou dans $t$. Posons $\Sigma = \{ (j, \epsilon), \; 1 \leq j \leq |\sequent| \}$, et considérons l'ensemble $\highapprox$ calculé par APPROX$(t, a', \Sigma, \Sigma)$.
    On a $\highapprox = \highapproxspec \left( t, \; \sequent, \; a' \right)$.
\end{theorem}

\begin{remark}
    L'invariant qu'on prouvera dans la démonstration vient de la remarque~\ref{algo_expl}, qui explique également pourquoi l'appel initial se fait sur l'ensemble $\Sigma$: pour une représentation réduite à un trou, l'ensemble $\highapprox'$ est celui des adresses racines du séquent.
\end{remark}

\begin{remark}
    Dans toute la suite, on notera désormais $\highapprox \left( t, \; \sequent, \; a' \right)$ et $\lowapprox \left( t, \; \sequent, \; a' \right)$ les résultats de l'appel à APPROX $(t, a', \Sigma, \Sigma)$. Cette notation ne pose pas problème, puisque $\Sigma$ est entièrement déterminé par la donnée de $\sequent$.
\end{remark}

Avant de donner la démonstration, définissons une notion qui lui sera utile:

\begin{definition}[Simplification d'un arbre]
    Soient $t \in \representationspartial$, $a' \in \treeaddresses$, tels que $a'$ adresse un noeud $N$ de $t$. La simplification de $t$ en $a'$, notée $\treesimplify(t, \; a')$, est l'arbre $t$ dans lequel on a remplacé la branche enracinée en $N$ par un trou. 
\end{definition}

\begin{remark}
    D'après le théorème~\ref{approxofcorrect}, on a $\treesimplify(t, \; a') \in \representationspartial$.
\end{remark}

\begin{example}
    Un arbre et sa simplification en $\Left$ (la simplification est à gauche):

        \begin{equation*}
        \begin{tikzpicture}%
            [level 2/.style={sibling distance=3.5cm}]
            \node {$U_{1 \epsilon}$}
            child {node {$\unknown$}
            };
        \end{tikzpicture}
        \relapproxlarge
        \begin{tikzpicture}%
            [level 2/.style={sibling distance=3.5cm}]
            \node {$U_{1 \epsilon}$}
            child {node {$B_{1 \Right}$}
                child {node {$F_{1 \Left, \; 1 \Right \cdot \Left}$}}
                child {node {$F_{1 \Right \cdot \Right, \; 2 \epsilon}$}}
            };
        \end{tikzpicture}
    \end{equation*}
\end{example}

\begin{proof}
    Montrons une propriété plus générale. Au fil des appels récursifs, on conserve un invariant: au début du $i$\ieme{} appel à APPROX$(t_i, a'_i, {\highapprox}^i, {\lowapprox}^i)$, en posant $a' = {\alpha}_i \cdot a'_i$, on a que $t_i$ est un sous-arbre de $t$, dont la racine est adressée par ${\alpha}_i$ dans $t$. De plus, $a'_i$ adresse dans $t_i$ le trou adressé par $a'$ dans $t$, et ${\highapprox}^i = \highapproxspec \left( \treesimplify( t, \; {\alpha}_i ), \sequent, \; {\alpha}_i \right)$.

    La preuve de cet invariant est en démonstration annexe~\ref{highapprox_correction_proof}. On en déduit, ainsi que de la terminaison, la correction de notre algorithme: il termine, et se retrouve donc forcément dans le cas où $a'_i = \epsilon$. Or, à ce moment-là, $\alpha_i = a'$, et, d'après l'invariant, ${\highapprox}^i = \highapproxspec \left( \treesimplify( t, \; {\alpha}_i ), \sequent, \; {\alpha}_i \right)$. L'algorithme renvoie donc ${\highapprox}^i$, d'où $\highapprox = \highapproxspec \left( \treesimplify( t, \; a' ), \sequent, \; a' \right)$. Comme $a'$ adresse un trou dans $t$, on a $\treesimplify( t, \; a' ) = t$, et ainsi $\highapprox = \highapproxspec \left( t, \sequent, \; a' \right)$.
\end{proof}

Spécifions maintenant l'approximation basse, en ne considérant qu'un cas particulier. Cette spécification restreinte suffira à justifier la correction de notre assistant, et nous nous en contenterons. Pour justifier la complétude, il nous faudra en effet recourir à un autre type de propriété sur les approximations, différent des spécifications.
    
\begin{theorem}[Propriété de branche de l'approximation basse]
    \label{lowapprox_correction}
    
    Soient $(t, \; \sequent) \in \representationspartial$ et $a' \in \treeaddresses$, tels que $t$ a un unique trou adressé par $a'$.
    On a, dans ce cas, $\lowapprox = \highapproxspec \left( t, \; \sequent, \; a' \right)$. 
\end{theorem}

\begin{proof}
    On peut simplement montrer qu'on a, dans ce cas, $\lowapprox=\highapprox$, et se servir du théorème \ref{highapprox_correction}. Cela se montrerait à nouveau par un invariant, de manière très directe. 
    
    En effet, l'arbre ayant un unique trou, on rentre toujours dans le cas $t_2 \in \trees$ du constructeur $B$, et on vérifie donc aisément que les opérations récursives effectuées sur $\lowapprox$ et $\highapprox$ sont toujours les mêmes.
\end{proof}

\subsection{Correction de l'assistant de preuve}
\label{correction_section}

Les résultats que nous venons de prouver servent à justifier une correction de notre assistant.

Lors de l'interaction, celui-ci impose, comme on l'a dit, des contraintes à l'utilisateur : il ne peut piocher que parmi les n\oe uds qu'on lui donne, et a, de plus, des n\oe uds obligatoires à utiliser dans certains cas. La suite d'approximations immédiates qu'il construit est donc particulière, et on cherche maintenant à la décrire:


\begin{definition}[Suite exacte]
    Soient un séquent $\sequent \in \sequents$, et $t_0 \relapprox t_1 \relapprox \cdots \relapprox t_n$ une suite d'approximations immédiates, telles que $t_0 = \unknown$ et $t_n \in \trees$.

    On dit que $(t_i)$ est une suite exacte selon $\sequent$, ssi les deux conditions suivantes sont vérifiées:
    \begin{enumerate}
        \item[$\exactcond$:]
        Pour tout $0 \leq i \leq n-1$, si l'on a ajouté à $t_i$ un noeud (et pas une feuille) étiqueté par $a$, alors l'arité de ce noeud correspond à celle de la sous-formule de $\sequent$ adressée par $a$, et $a \in \highapprox(t_i, \; \sequent, \; \beta)$, où $\beta$ est l'adresse du trou de $t_i$ qu'on a remplacé. Si l'on ajouté une feuille étiquetée par $a_1$ et $a_2$, alors $a_1$ et $a_2$ adressent des atomes duaux de $\sequent$, et $a_1, a_2 \in \highapprox(t_i, \; \sequent, \; \beta)$.

        \item[$\exactcondbis$:]
        Pour tout $0 \leq i \leq n-1$, si l'on a ajouté à $t_i$ une feuille $F$ étiquetée par $a_1$ et $a_2$, alors $\lowapprox(t_i, \; \sequent, \; \beta) \subseteq \{a_1, \; a_2\}$, où $\beta$ est l'adresse du trou de $t_i$ qu'on a remplacé.
    \end{enumerate}
\end{definition}

\begin{remark}
    La condition $\exactcond$ dit que l'utilisateur pioche toujours dans l'ensemble $\highapprox$ qu'on lui donne.
    
    La condition $\exactcondbis$ dit qu'il a utilisé tous les n\oe uds obligatoires sur une branche au moment où l'on peut le vérifier, c'est-à-dire au moment où l'on ferme la branche en appliquant un axiome. Cette condition est liée à $\lowapprox$.
\end{remark}

\begin{remark}
    \label{correction_desc}
    La condition $t_n \in \trees$ ne nous fait considérer que les cas où l'utilisateur a abouti, puisqu'il doit avoir fermé l'arbre. C'est la correction la plus forte que l'on puisse donner.
    
    En effet, on ne peut rien dire des cas où l'arbre n'a pas encore été fermé, car l'utilisateur peut s'être déjà trompé, ou bien le séquent peut ne pas être prouvable, mais nos conditions de correction partielles ne permettent pas toujours de le savoir. Cela vient du fait que la réciproque du théorème~\ref{approxofcorrect} est fausse.

    \begin{example}
        Un contre-exemple à ce théorème:
\quad
        $F(1 \epsilon, \; 2 \Left) \vdash X_1, {X_1}\orth \tensor {X_2}\orth, X_2$

        Cette représentation, vue comme partielle, est correcte, mais n'est pourtant pas l'approximation d'une représentation correcte complète. En effet, elle n'est pas correcte si on la voit comme une représentation complète (la linéarité \ref{clin} est violée), et on ne peut pas non plus la prolonger.
    \end{example}
    
    Grâce à nos contraintes, nous pouvons ainsi éviter certains cas d'erreur, mais nous ne pouvons garantir que tous les choix que nous autorisons permettent d'aboutir.  
    
    En revanche, si le séquent est prouvable, il existera toujours une manière d'aboutir en partant de l'arbre trou, comme nous le verrons en justifiant la complétude.
\end{remark}

\begin{implementation}
    Les conditions $\exactcond$ et $\exactcondbis$ sont toujours maintenues par notre code, grâce à des vérifications faites quand l'utilisateur donne ses choix.

    En effet, lorsque l'utilisateur fait un choix violant l'une de nos deux conditions, le code revient en arrière et n'accepte pas son opération. Une message d'erreur est affiché, et il lui est demandé de faire un autre choix. 
    
    Notons qu'il est possible de se retrouver bloqué, si plus aucun choix autorisé n'est bon. Cela se produit lorsqu'il est devenu impossible d'aboutir, à cause d'erreurs précédentes non détectées, ou d'une non prouvabilité du séquent, que nous ne pouvons pas détecter non plus.
\end{implementation}

Dans les cas où l'utilisateur a abouti, cependant, il a construit une suite exacte selon le séquent qu'il cherche à prouver, et nous pouvons alors lui garantir la correction de sa preuve: 

\begin{theorem}[Correction de l'assistant]
    Soient un séquent $\sequent \in \sequents$, et $t_0 \relapprox t_1 \relapprox \cdots \relapprox t_n$ une suite exacte selon $\sequent$.
    On a $(t_n, \sequent) \in \representations$.
\end{theorem}

\begin{remark}
    Si l'utilisateur aboutit, il a donc construit une représentation correcte, qu'on peut décoder et afficher en tant que preuve. C'est ce qui est fait par notre code.
\end{remark}

\begin{proof}
    D'après la condition $\exactcond$ et la spécification de $\highapprox$ (théorème~\ref{highapprox_correction}), on a, par récurrence immédiate, $(t_i, \sequent) \in \representationspartial, \; \; \forall 0 \leq i \leq n$.
    
    En effet, à chaque fois qu'on pioche une adresse dans $\highapprox$, et qu'on s'en sert pour substituer un trou, la spécification $\highapproxspec$ garantit que l'adresse que l'adresse est correcte. En associant ce résultat à la condition $\exactcond$, on déduit que le bon adressage \caddpartial{} est conservé. Par ailleurs, la spécification $\highapproxspec$ garantit que les adresses utilisées n'apparaissaient pas dans l'arbre qu'on a modifié, et donc que la sous-linéarité \clinpartial{} est conservée. Enfin, toujours d'après la spécification $\highapproxspec$, les adresses utilisées descendent immédiatement d'une adresse présente sur la branche racine--trou, en respectant les conditions gauche et droite. Cela assure que la descendance \cdespartial{} est conservée. Les représentations partielles sont donc toutes dans $\representationspartial$.

    Supposons par l'absurde qu'on ait $(t_n, \sequent) \notin \representations$. Comme $(t_n, \sequent) \in \representationspartial$, il existe nécessairement des adresses de sous-formules de $\sequent$ n'apparaissant pas dans $t_n$. Choisissons une telle adresse $a \in \addresses$ minimale (au sens de la remarque~\ref{min_address}). On a deux cas:

    \begin{description}
        \item[$\bm{a = (m, \epsilon)}$] $t_{n-1}$ contient un unique trou, et celui-ci est nécessairement remplacé par une feuille, car $t_n \in \trees$. Notons $a_1$ et $a_2$ les adresses de cette feuille, et $\beta$ l'adresse du trou.
        
        $a$ est une adresse racine correcte n'apparaissant pas dans $t$. D'après le théorème~\ref{lowapprox_correction}, $t$ n'ayant qu'un trou adressé par $\beta$, on a donc $a = (m, \epsilon) \in \lowapprox(t_i, \sequent, \; \beta)$, d'où $\lowapprox(t_i, \sequent, \; \beta) \nsubseteq \{a_1, \; a_2\}$.
        Ceci contredit la condition $\exactcondbis$, c'est absurde.

        \item[$\bm{a = (m, \someadd \cdot \alpha)}$] $a$ étant une adresse non utilisée minimale, $a' = (m, \someadd)$ apparaît nécessairement dans $t_n$. Notons $\beta$ l'adresse du n\oe ud étiqueté par $a'$.

        Comme la suite d'approximations immédiates $t_0 \relapprox t_1 \relapprox \cdots \relapprox t_n$ commence en $H$ et termine en $t_n \in \trees$, on peut en extraire une sous-suite $t'_0 \relapprox \cdots \relapprox t'_m$, où $t'_0$ est le dernier arbre ne contenant pas l'adresse $a' = (m, \someadd)$, et $t'_m$ est le premier arbre dont le sous-arbre enraciné en $\beta$ est complet.

        En extrayant de ces arbres les sous-arbres enracinés en $\beta$, on obtient une nouvelle suite d'approximations immédiates commençant en $H$, et terminant en un arbre complet. On peut alors modifier $\sequent$ pour ne conserver que les sous-formules mentionnées par l'arbre complet, puis ré-adresser tous les arbres de la suite pour que leurs adresses correspondent au nouveau séquent. On obtient ainsi un séquent $\sequent'$ et une suite d'approximations immédiates $t''_0 \relapprox \cdots \relapprox t''_m$ vérifiant $t''_0 = H$, $t''_m \in \trees$, ainsi que les conditions $\exactcond$ et $\exactcondbis$.

        Comme dans le cas précédent, on peut désormais considérer $t''_{m-1}$ qui contient un unique trou, remplacé dans $t''_m$ par une feuille. La racine de $t''_{m-1}$ est le ré-adressage de $a' = (m, \rho)$, de la forme $(m', \epsilon)$. De plus, le ré-adressage de $a$, qui est une adresse correcte de $\sequent'$ n'apparaissant pas dans dans $t''_{m-1}$, est de la forme $(m', \alpha)$, et descend donc immédiatement d'une adresse de la branche racine--trou.

        D'après le théorème~\ref{lowapprox_correction}, $a'$ appartient donc à l'ensemble $\lowapprox$ calculé au moment où l'on remplace le trou de $t''_{m-1}$ par une feuille, ce qui est absurde, car cela viole à nouveau la condition $\exactcondbis$.
\qedhere
    \end{description}
\end{proof}

\subsection{Complétude de l'assistant de preuve}

Justifions maintenant une complétude, qui garantit que, si notre assistant n'empêche pas l'utilisateur de se tromper, il lui permet néanmoins, s'il ne se trompe pas, de construire n'importe quelle représentation correcte, et donc, en un certain sens, n'importe quelle preuve (voir la remarque~\ref{remark_kernel}).

\begin{theorem}[Complétude de l'assistant de preuve]
    \label{completeness}
    Soient $t \in \trees, \; \sequent \in \sequents$, tels que $(t, \; \sequent) \in \representations$.
    Il existe une suite d'approximations immédiates $t_0 \relapprox \cdots \relapprox t_n$, exacte selon $\sequent$, telle que $t_n = t$.
\end{theorem}

\begin{remark}
    Cette suite peut être construite à l'aide de l'assistant, puisqu'elle respecte les contraintes imposées lors de l'interaction.
\end{remark}

\begin{proof}
    Par récurrence forte sur la taille de $t$. On regarde le n\oe ud présent à la racine:
    \begin{description}
        \item[$F:$] 
            Si $(t, \sequent) = F(n\rho, n'\rho') \vdash \sequent \in \representations$.
        
            De la même manière que dans la preuve du théorème~\ref{rep_inversion}, on a $t = F(1 \epsilon, \; 2 \epsilon)$, et $\sequent = X\orth, X$ ou $\sequent = X, X\orth$, avec $X$ un atome. Dans les deux cas, la suite $H \relapprox \; F(1 \epsilon, \; 2 \epsilon)$ convient. En effet, celle-ci est une suite d'approximations immédiates, et les deux conditions d'exactitude par rapport à $\sequent$ sont vérifiées:
    
            \begin{enumerate}
                \item[$\exactcond$:] On a ajouté une feuille, étiquetée par $1 \epsilon$ et $2 \epsilon$. Ces adresses étiquettent bien deux atomes duaux, et, de plus, d'après le théorème~\ref{highapprox_correction} de spécification de l'approximation haute, on a $\highapproxspec = \{ 1 \epsilon$, $2 \epsilon\}$. La condition $\exactcond$ est bien vérifiée.
    
                \item[$\exactcondbis$:] L'ensemble des sous-adresses de $\sequent$ est $\{ 1 \epsilon, 2 \epsilon \}$. De plus, $H$ n'a qu'un seul trou, d'où, d'après le théorème~\ref{lowapprox_correction} spécifiant $\lowapprox$, la propriété d'inclusion à vérifier pour $\exactcondbis$ ne peut être violée.
            \end{enumerate}

        \item[$B$:] 
            Si $(t, \sequent) = B(n\rho, t_1, t_2) \vdash \sequent \in \representations$.

            De même que dans la preuve du théorème~\ref{rep_inversion}, $\rho = \epsilon$, et, en reprenant les mêmes notations, on construit les représentations correctes $\left( \psi_1 \left( t_1 \right), \; \sequentbis_1 \right)$ et $\left( \psi_2 \left( t_2 \right), \; \sequentbis_2 \right)$. On se donne de plus une permutation $\someperm$, telle que $\permapp{\someperm}{(G, F_1 \tensor F_2, D)} = \sequent_1, F_1 \tensor F_2, \sequent_2$. Rappelons que $\sequentbis_1 = G, F_1$, et $\sequentbis_2 = F_2, D$.
    
            Par hypothèses de récurrence sur $\left( \psi_1 \left( t_1 \right), \; \sequentbis_1 \right)$ et $\left( \psi_2 \left( t_2 \right), \; \sequentbis_2 \right)$, il existe deux suites d'approximations immédiates $g_0 = H \relapprox \cdots \relapprox g_{m_1} = \psi_1 \left( t_1 \right)$ et $d_0 = H \relapprox \cdots \relapprox d_{m_2} = \psi_2 \left( t_2 \right)$ respectivement exactes par rapport à $\sequentbis_1$ et $\sequentbis_2$.
    
            On construit alors une suite d'approximations immédiates $(t'_i):$
            \begin{align*}
                \begin{multlined}
                    \unknown
                    \relapprox
                    \begin{tikzpicture}%
                    [level 1/.style={sibling distance=3.5cm}]
                    \node {$B_{n \epsilon}$}
                        child {node {$\someperm \left( \psi_{\tensor\Left} \left(
                        g_0\right) \right) = \unknown$}}
                        child {node {$\someperm \left( \psi_{\tensor\Right} \left(
                        d_0\right) \right) = \unknown$}
                    };
                    \end{tikzpicture}
                    \relapprox \cdots \relapprox
                    \begin{tikzpicture}%
                    [level 1/.style={sibling distance=3.5cm}]
                    \node {$B_{n \epsilon}$}
                        child {node {$\someperm \left( \psi_{\tensor\Left} \left(
                        g_{m_1}\right) \right)$}}
                        child {node {$\someperm \left( \psi_{\tensor\Right} \left(
                        d_0\right) \right) = \unknown$}
                    };
                    \end{tikzpicture}\\
                    \relapprox \cdots \relapprox
                    \begin{tikzpicture}%
                    [level 1/.style={sibling distance=3.5cm}]
                    \node {$B_{n \epsilon}$}
                        child {node {$\someperm \left( \psi_{\tensor\Left} \left(
                        g_{m_1}\right) \right)$}}
                        child {node {$\someperm \left( \psi_{\tensor\Right} \left(
                        d_{m_2}\right) \right)$}
                    };
                    \end{tikzpicture}
                    =
                    \begin{tikzpicture}%
                    [level 1/.style={sibling distance=3.5cm}]
                    \node {$B_{n \epsilon}$}
                        child {node {$\someperm \left( \psi_{\tensor\Left} \left( \psi_1 \left(
                            t_1
                          \right) \right) \right)$}}
                        child {node {$\someperm \left( \psi_{\tensor\Right} \left( \psi_2 \left(
                            t_2
                          \right) \right) \right)$}
                    };
                    \end{tikzpicture}\\
                \end{multlined}
            \end{align*}
    
            Chaque élément de cette suite est bien une approximation immédiate de son successeur, car $(g_i)$ et $(d_i)$ sont des suites d'approximations immédiates, et le fait d'être une approximation immédiate est stable par ré-adressage et ajout de contexte (cela se vérifie facilement).
    
            De plus, comme montré dans la preuve du théorème~\ref{lowapprox_correction}, $\someperm \circ \psi_{\tensor\Left} \circ \psi_1 = \textit{Id}$, et $\someperm \circ \psi_{\tensor\Right} \circ \psi_2 = \textit{Id}$. La suite $(t_i)$ part donc bien de $H$, et termine en $t$.
            Il reste à montrer que cette suite $(t_i)$ est exacte selon $\sequent$.
            Cela se prouve par des propriétés de contexte, données en annexe (théorème~\ref{context_prop}). Elles disent essentiellement que le fait de contextualiser une suite exacte, comme nous l'avons fait ici en ajoutant une racine et un voisin droit ou gauche, préserve l'exactitude. 

        \item[$U$:]
            Si $t_i = U(n\rho, t') \in \representations$.
            
            L'argument est similaire au cas du constructeur $B$.
\qedhere
    \end{description}
\end{proof}

\section{Complétion automatique et détection d'erreurs}

Lors de l'interaction avec l'utilisateur, il arrive parfois qu'on soit capable de compléter automatiquement la représentation qu'il est en train de construire, sans lui retirer de contrôle dessus. Cela se produit lorsque le choix qu'il peut faire est unique, et qu'on peut donc l'appliquer à sa place afin de lui faire gagner du temps.

Par ailleurs, certains cas d'erreurs, non détectés par nos contraintes sur les représentations partielles correctes, peuvent être repérés par d'autres vérifications que nous donnerons, et qui permettent de renforcer la correction que nous avons justifiée.

\subsection{Complétion automatique}

Donnons une définition formelle pour les autocomplétions, qui correspond à ce que l'on a dit en préambule:

\begin{definition}[Fonction d'autocomplétion]
$f: \representationspartial \rightarrow \representationspartial$ est une fonction d'autocomplétion ssi:
\begin{equation*}
    \forall t' \in \representationspartial, \; \forall t \in \representations, \; t' \relapproxlarge t \implies f(t') \relapproxlarge t.
\end{equation*}
\end{definition}

\begin{remark}
    Cette définition dit précisément qu'une autocomplétion ne retire à l'utilisateur aucun contrôle sur la construction de sa preuve. On peut facilement la mettre en lien avec ce qui a été dit pour la justification de la complétude. 
\end{remark}

\begin{remark}
    Remarquons qu'on utilise la relation d'approximation large, puisque notre autocomplétion peut modifier l'ordre dans lequel l'utilisateur aurait fait ses opérations. Cela n'a pas beaucoup d'importance, puisque la représentation obtenue à l'issue reste la même.
\end{remark}

Listons brièvement les heuristiques que nous utilisons dans notre code. Il est assez clair, au vu des spécifications des approximations basse et haute, qu'elles sont des autocomplétions. Les démonstrations ne seront donc pas détaillées. Notons qu'il serait possible, si l'on trouvait d'autres heuristiques, de les ajouter à celles déjà présentes, sans que cela ne pose problème.

\begin{implementation}
    L'autocomplétion est implémentée dans notre code par:

    \begin{minted}{ocaml}
        let atom_auto_complete t s a = ...
        let rec autocomplete t s  ...
    \end{minted}
    
    La fonction autocomplete se sert de atom\_auto\_complete, et ces fonctions ont été conçues afin de pouvoir être modifiées et incrémentées.
\end{implementation}

\begin{itemize}
    \item Si, pour un trou, seuls des atomes peuvent être choisis, et qu'un seul d'entre eux est obligatoire et a un dual, ou si seuls deux d'entre eux sont obligatoires et duaux, on applique une feuille avec ces deux atomes.

    \begin{remark}
        L'idée derrière cette heuristique est que les deux atomes duaux doivent forcément être utilisés dans la branche, et qu'ils ne pourront pas l'être à un autre moment, puisqu'il n'y a plus d'opérateur possible.
    \end{remark}

    \item Si, pour un trou, seuls des atomes à l'exception d'un unique opérateur peuvent être choisis, et que l'opérateur est obligatoire, ou qu'aucun des atomes n'a de paire duale, on applique l'opérateur.

    \begin{remark}
        Dans le cas où l'opérateur est obligatoire, on doit l'utiliser à ce moment car tout autre choix induirait l'utilisation d'un axiome, et ainsi la fermeture de la branche. Dans le cas où aucun atome n'a de paire duale, on ne peut qu'appliquer l'opérateur, ce qu'on fait, car on sait qu'on devra fermer la branche.
    \end{remark}
\end{itemize}

Dans notre implémentation, à chaque fois que l'utilisateur fait un choix, la représentation résultante est autocomplétée itérativement, jusqu'à ce que ce ne soit plus possible. Pour ne pas perdre l'utilisateur, chaque autocomplétion est affichée, avec le cas qui l'a engendrée.

Si, au cours des autocomplétions, on viole l'une de nos deux conditions sur la correction partielle (voir section~\ref{correction_section}), cela veut dire que l'utilisateur a fait un mauvais choix, puisqu'il aurait été obligé, en continuant, de faire l'opération incorrecte que nous avons faite à sa place. Nous revenons donc complètement en arrière, et rejetons son choix.

Cela enrichit la correction, au sens de la remarque~\ref{correction_desc}. En effet, nous ne garantissons toujours pas que chaque choix permet d'aboutir, mais nous éliminons plus de choix incorrects. Nous facilitons ainsi le travail de l'utilisateur, en lui évitant plus souvent de se tromper, mais nous ne pouvons toujours pas montrer de correction ``forte''.

\subsection{Détection d'erreurs}

De même, listons quelques heuristiques pour la détection d'erreurs. À nouveau, les preuves ne seront pas détaillées, et ces heuristiques pourraient également être enrichies sans que ça ne pose problème.

\begin{implementation}
    La majorité des heuristiques sont implémentées dans notre code par:

    \begin{minted}{ocaml}
        let atom_auto_complete t s a = ...
    \end{minted}
    
    Le rattrapage des erreurs, qui permet de rejeter le choix de l'utilisateur, et de revenir en arrière en affichant un message d'erreur, est implémenté par:

    \begin{minted}{ocaml}
        let prove_sequent s = ...
    \end{minted}
\end{implementation}

\begin{itemize}
    \item Si, pour un trou, seuls des atomes peuvent être choisis, et qu'un atome obligatoire ne possède pas de dual parmi les autres choix possibles, on renvoie une erreur.

    \begin{remark}
        Cet atome ne pourra, en effet, jamais être utilisé dans la branche, puisqu'il ne peut l'être qu'avec la règle d'axiome, qui impose de l'associer à un dual.
    \end{remark}

    \item Si, pour un trou, seuls des atomes peuvent être choisis, et qu'au moins deux atomes non duaux sont obligatoires, on renvoie une erreur.

    \begin{remark}
        En effet, on ne pourra ajouter dans la branche qu'un axiome, qui la fermera, et l'atome restant ne pourra pas être utilisé.
    \end{remark}
\end{itemize}

Lorsqu'une de ces erreurs est détectée, cela produit le même effet que quand une de nos deux conditions de correction partielle est violée: on revient en arrière, et on rejette le choix de l'utilisateur. Ces heuristiques, combinées à l'autocomplétion, permettent à nouveau d'éliminer plus de mauvais choix, et donc, en un certain sens, d'enrichir la correction.

\section{Conclusion}

Notre assistant de preuve offre, conformément à notre objectif, une solution plus pratique que Click \& Collect pour les utilisateurs experts. Si le lecteur veut s'en convaincre, il peut comparer les deux outils sur des exemples, en construisant des preuves. L'exemple~\ref{exconcl} de notre code est particulièrement parlant, et on peut également le manipuler, \href{https://click-and-collect.linear-logic.org/?s=X5%2C+X4%2C+X1%5E+*+%28X2%5E+*+%28X3%5E+*+%28X4%5E+*+X5%5E%29%29%29%2C+X1%2C+X3+%7C+X2}{sur ce lien}, dans Click \& Collect.

\begin{example}\label{exconcl}
    L'exemple en question:
%
    $X_5, X_4, {X_1}\orth \tensor ({X_2}\orth \tensor ({X_3}\orth \tensor ({X_4}\orth \tensor {X_5}\orth))), X_1, X_3 \parr X_2$

    Les suites d'instructions $5$ ou $(3, 2, 5)$ permettent par exemple de conclure avec notre prouveur. C'est plus rapide que dans Click \& Collect.
\end{example}

La correction et la complétude que nous avons prouvées garantissent la fiabilité de notre assistant. Les heuristiques d'autocomplétion et de détection d'erreurs rendent l'expérience de l'utilisateur plutôt fluide et agréable. Celle-ci pourrait cependant être améliorée en ajoutant une interface graphique, et en permettant à l'utilisateur de revenir en arrière dans ses choix, afin qu'il ne se retrouve pas bloqué par une erreur non détectée.

De plus, l'autocomplétion pourrait être encore enrichie, en étiquetant directement les trous des représentations par leurs approximations d'adresses. Ceci permettrait d'avoir des approximations plus fines, en utilisant un algorithme global. Celui-ci prendrait en entrée une représentation partielle, et mettrait à jour ses étiquettes d'approximations de manière itérative. Les approximations ne seraient donc plus calculées individuellement, pour un trou donné, mais en même temps, et les unes par rapport aux autres. Elles seraient ainsi plus précises, et cela permettrait de rentrer plus souvent dans des cas d'autocomplétion, et donc de détecter plus d'erreurs.

Enfin, nous n'avons considéré que le fragment multiplicatif MLL de la logique linéaire, mais notre assistant pourrait être étendu, en utilisant la même méthode, aux fragments plus grands, comme par exemple LL.

\printbibliography



\clearpage

\appendix

\section{Contexte institutionnel}
\label{inst_contex}
Ce stage a été effectué au LIP (Laboratoire de l'Informatique et du Parallélisme), au sein de l'équipe Plume. 

Ce laboratoire a été créé à l’École Normale Supérieure de Lyon en 1988, et accueille une soixantaine de chercheurs permanents, ainsi qu'une quarantaine de doctorants. Il comporte huit équipes (Aric, Avalon, Cash, Ockham, Hownet, MC2, Plume et Roma), travaillant sur des domaines variés de l'informatique théorique et des sciences de l'information. 

L'équipe Plume travaille plus particulièrement sur la théorie de la preuve et la sémantique formelle, en cherchant à fournir des méthodes pour prouver la correction des programmes informatiques. Deux types de méthodes y sont étudiées: les méthodes \textit{a priori}, dites ``par construction'', et les méthodes \textit{a posteriori}, dites ``par vérification''. Les outils utilisés vont ainsi de Coq à la théorie des catégories, en passant par les bases de données orientées graphe et la logique linéaire, sujet de ce stage. La logique linéaire s'inscrit dans les méthodes dites ``par construction''.

Au cours de mon stage, j'ai été encadré par Olivier Laurent, directeur de recherche au CNRS et chercheur permanent au LIP. J'ai eu l'occasion de participer aux groupes de travail et conférences hebdomadaires organisés par l'équipe Plume, ainsi qu'à la rencontre nationale CHoCoLa de juin. J'ai également pu discuter et interagir avec des doctorants et chercheurs, dont Samuel Arsac, Amélie Rima, Le Thanh Dung ‘Tito’ Nguyen, Denis Kuperberg et Daniel Hirschkoff.

\section{Une trace d'exécution}
\label{exec_trace}
Décrivons l'exécution de notre assistant dans un cas où l'on a prouvé le séquent $X_1 \tensor {X_2}\orth, X_2 \tensor {X_3}\orth, {X_1}\orth \parr X_3$.

Le programme affiche tout d'abord un pseudo-code \LaTeX{}, associé à la représentation partielle minimale $\unknown \vdash X_1 \tensor {X_2}\orth, X_2 \tensor {X_3}\orth, {X_1}\orth \parr X_3$. Ce pseudo-code est celui d'un arbre de preuve partiel, où les trous de la représentation apparaissent comme des hypothèses. Celles-ci portent des séquents, calculés à partir des approximations basse et haute: les sous-formules affichées correspondent aux adresses de l'ensemble $\highapprox$, et les formules liées à l'ensemble $\lowapprox$ y sont surlignées par les caractères $<>$. 

Ici, l'arbre est réduit à un trou, ce qui explique que toutes les formules sont obligatoires:

\begin{minted}{shell-session}
\hypv{<(1) * (2^)>, <(2) * (3^)>, <(1^) | (3)>}
\end{minted}

Comme il n'y a qu'un seul trou, il est sélectionné automatiquement:

\begin{minted}{shell-session}
Unique hole 1 automatically selected
\end{minted}

Le séquent lié au trou est ensuite affiché:

\begin{minted}{shell-session}
Selected hole:
  <(1) * (2^)>, <(2) * (3^)>, <(1^) | (3)>
Please choose the rule to apply:
\end{minted}

L'utilisateur choisit la règle qu'il applique, en donnant un entier correspondant à l'indice d'une formule dans le séquent (numérotées de gauche à droite à partir de 1):

\begin{minted}{shell-session}
3
\end{minted}

La règle est appliquée par l'assistant, et on obtient la représentation partielle suivante:

\begin{equation*}
    \begin{tikzpicture}%
        [level 2/.style={sibling distance=3.5cm}]
        \node {$U_{3 \epsilon}$}
        child {node {$\unknown$}
        };
    \end{tikzpicture} \vdash X_1 \tensor {X_2}\orth, X_2 \tensor {X_3}\orth, {X_1}\orth \parr X_3
\end{equation*}

L'arbre partiel associé à cette représentation est affiché:

\begin{minted}{shell-session}
   \hypv{<(1) * (2^)>, <(2) * (3^)>, <1^>, <3>}
\parrv{<(1) * (2^)>, <(2) * (3^)>, <(1^) | (3)>}
\end{minted}

À nouveau, cet arbre ne contient qu'un seul trou, qui est automatiquement sélectionné:

\begin{minted}{shell-session}
Unique hole 1 automatically selected
Selected hole:
  <(1) * (2^)>, <(2) * (3^)>, <1^>, <3>
Please choose the rule to apply:
\end{minted}

L'utilisateur fait un nouveau choix:
\begin{minted}{shell-session}
1
\end{minted}

On obtient la représentation partielle suivante:
\begin{equation*}
    \begin{tikzpicture}%
        [level 2/.style={sibling distance=3.5cm}]
        \node {$U_{3 \epsilon}$}
        child {node {$B_{1 \epsilon}$}
            child {node {$\unknown$}}
            child {node {$\unknown$}}
        };
    \end{tikzpicture} \vdash X_1 \tensor {X_2}\orth, X_2 \tensor {X_3}\orth, {X_1}\orth \parr X_3
\end{equation*}

Et on l'affiche:

\begin{minted}{shell-session}
      \hypv{<1>, (2) * (3^), 1^, 3}

      \hypv{<2^>, (2) * (3^), 1^, 3}
   \tensorv{<(1) * (2^)>, <(2) * (3^)>, <1^>, <3>}
\parrv{<(1) * (2^)>, <(2) * (3^)>, <(1^) | (3)>}
\end{minted}
Elle correspond à la pseudo-preuve partielle suivante (``pseudo'' dans la mesure où les règles d'échange n'apparaissent pas) munie des approximations basse et haute (les formules en rouge sont les formules obligatoires, les autres sont les formules possibles):
\begin{prooftree*}
      \hypv{\foblig{X_1}, X_2\tensor X_3\orth, X_1\orth, X_3}

      \hypv{\foblig{X_2\orth}, X_2\tensor X_3\orth, X_1\orth, X_3}
   \tensorv{\foblig{X_1\tensor X_2\orth}, \foblig{X_2\tensor X_3\orth}, \foblig{X_1\orth}, \foblig{X_3}}
\parrv{\foblig{X_1\tensor X_2\orth}, \foblig{X_2\tensor X_3\orth}, \foblig{X_1\orth\parr X_3}}
\end{prooftree*}

Ici, on rentre dans un cas d'autocomplétion: le trou de droite ne peut être remplacé que par des atomes, à l'exception d'un tenseur. Or, aucun atome n'a de paire duale. On en déduit qu'on doit appliquer le tenseur:

\begin{minted}{shell-session}
Quasi-atomic sequent <2^>, (2) * (3^), 1^, 3 with a unique mandatory (or possible)
tensor operator: applying it
\end{minted}

On obtient la représentation partielle suivante:

\begin{equation*}
    \begin{tikzpicture}%
        [level 2/.style={sibling distance=3.5cm}]
        \node {$U_{3 \epsilon}$}
        child {node {$B_{1 \epsilon}$}
            child {node {$\unknown$}}
            child {node {$B_{2 \epsilon}$}
                child{node {$\unknown$}}
                child{node {$\unknown$}}}
        };
    \end{tikzpicture} \vdash X_1 \tensor {X_2}\orth, X_2 \tensor {X_3}\orth, {X_1}\orth \parr X_3
\end{equation*}

Et on l'affiche:

\begin{minted}{shell-session}
      \hypv{<1>, 1^, 3}

         \hypv{2^, <2>, 1^, 3}

         \hypv{2^, <3^>, 1^, 3}
      \tensorv{<2^>, <(2) * (3^)>, 1^, 3}
   \tensorv{<(1) * (2^)>, <(2) * (3^)>, <1^>, <3>}
\parrv{<(1) * (2^)>, <(2) * (3^)>, <(1^) | (3)>}
\end{minted}
Elle correspond à la pseudo-preuve partielle suivante munie des approximations basse et haute:
\begin{prooftree*}
      \hypv{\foblig{X_1}, X_1\orth, X_3}
         \hypv{X_2\orth, \foblig{X_2}, X_1\orth, X_3}
         \hypv{X_2\orth, \foblig{X_3\orth}, X_1\orth, X_3}
      \tensorv{\foblig{X_2\orth}, \foblig{X_2\tensor X_3\orth}, X_1\orth, X_3}
   \tensorv{\foblig{X_1\tensor X_2\orth}, \foblig{X_2\tensor X_3\orth}, \foblig{X_1\orth}, \foblig{X_3}}
\parrv{\foblig{X_1\tensor X_2\orth}, \foblig{X_2\tensor X_3\orth}, \foblig{X_1\orth\parr X_3}}
\end{prooftree*}


On rencontre un nouveau cas d'autocomplétion: le trou de gauche ne peut être remplacé que par des atomes, et seul l'un d'entre eux est obligatoire et a une paire duale. On applique donc une feuille:

\begin{minted}{shell-session}
Atomic sequent <1>, 1^, 3 with a unique mandatory atom having a unique
complement: applying it
\end{minted}

On obtient la représentation partielle suivante:
\begin{equation*}
    \begin{tikzpicture}%
        [level 2/.style={sibling distance=3.5cm}]
        \node {$U_{3 \epsilon}$}
        child {node {$B_{1 \epsilon}$}
            child {node {$F_{1 \Left, \; 3 \Left}$}}
            child {node {$B_{2 \epsilon}$}
                child{node {$\unknown$}}
                child{node {$\unknown$}}}
        };
    \end{tikzpicture} \vdash X_1 \tensor {X_2}\orth, X_2 \tensor {X_3}\orth, {X_1}\orth \parr X_3
\end{equation*}

On l'affiche:

\begin{minted}{shell-session}
      \axv{<1>, <1^>}

         \hypv{2^, <2>, 3}

         \hypv{2^, <3^>, 3}
      \tensorv{<2^>, <(2) * (3^)>, <3>}
   \tensorv{<(1) * (2^)>, <(2) * (3^)>, <1^>, <3>}
\parrv{<(1) * (2^)>, <(2) * (3^)>, <(1^) | (3)>}
\end{minted}

On rencontre encore un cas d'autocomplétion, similaire:
\begin{minted}{shell-session}
Atomic sequent 2^, <2>, 3 with a unique mandatory atom having a unique
complement: applying it
\end{minted}

On obtient la représentation partielle suivante:
\begin{equation*}
    \begin{tikzpicture}%
        [level 2/.style={sibling distance=3.5cm}]
        \node {$U_{3 \epsilon}$}
        child {node {$B_{1 \epsilon}$}
            child {node {$F_{1 \Left, \; 3 \Left}$}}
            child {node {$B_{2 \epsilon}$}
                child{node {$F_{1 \Right, \; 2 \Left}$}}
                child{node {$\unknown$}}}
        };
    \end{tikzpicture} \vdash X_1 \tensor {X_2}\orth, X_2 \tensor {X_3}\orth, {X_1}\orth \parr X_3
\end{equation*}

On l'affiche:

\begin{minted}{shell-session}
      \axv{<1>, <1^>}

         \axv{<2^>, <2>}

         \hypv{<3^>, <3>}
      \tensorv{<2^>, <(2) * (3^)>, <3>}
   \tensorv{<(1) * (2^)>, <(2) * (3^)>, <1^>, <3>}
\parrv{<(1) * (2^)>, <(2) * (3^)>, <(1^) | (3)>}
\end{minted}

On tombe sur un dernier cas d'autocomplétion, toujours similaire:
\begin{minted}{shell-session}
Atomic sequent <3^>, <3> having only two mandatory and dual atoms: 
applying them
\end{minted}

On obtient alors une représentation complète:
\begin{equation*}
    \begin{tikzpicture}%
        [level 2/.style={sibling distance=3.5cm}]
        \node {$U_{3 \epsilon}$}
        child {node {$B_{1 \epsilon}$}
            child {node {$F_{1 \Left, \; 3 \Left}$}}
            child {node {$B_{2 \epsilon}$}
                child{node {$F_{1 \Right, \; 2 \Left}$}}
                child{node {$F_{2 \Right, \; 3 \Right}$}}}
        };
    \end{tikzpicture} \vdash X_1 \tensor {X_2}\orth, X_2 \tensor {X_3}\orth, {X_1}\orth \parr X_3
\end{equation*}

Le séquent a été prouvé:
\begin{minted}{shell-session}
Proven!
\end{minted}

La représentation complète est décodée, et le code en \LaTeX{} d'une preuve correspondante est affiché:

\begin{minted}{shell-session}
\begin{prooftree}
      \axv{{X_1}\orth, X_1}

         \axv{{X_2}\orth, X_2}

         \axv{{X_3}\orth, X_3}
      \tensorv{{X_2}\orth, (X_2) \tensor ({X_3}\orth), X_3}
   \tensorv{{X_1}\orth, (X_1) \tensor ({X_2}\orth), (X_2) \tensor 
   ({X_3}\orth), X_3}
   \permv{3 \;1 \;2 \;4}{(X_1) \tensor ({X_2}\orth), (X_2) \tensor 
   ({X_3}\orth), {X_1}\orth, X_3}
\parrv{(X_1) \tensor ({X_2}\orth), (X_2) \tensor ({X_3}\orth), ({X_1}\orth) \parr (X_3)}
\end{prooftree}
\end{minted}

C'est-à-dire la preuve:
\begin{prooftree*}
          \axv{{X_1}\orth, X_1}
    
             \axv{{X_2}\orth, X_2}
    
             \axv{{X_3}\orth, X_3}
          \tensorv{{X_2}\orth, X_2 \tensor {X_3}\orth, X_3}
       \tensorv{{X_1}\orth, X_1 \tensor {X_2}\orth, X_2 \tensor {X_3}\orth, X_3}
       \permv{(3, 1, 2, 4)}{X_1 \tensor {X_2}\orth, X_2 \tensor {X_3}\orth, {X_1}\orth, X_3}
    \parrv{X_1 \tensor {X_2}\orth, X_2 \tensor {X_3}\orth, {X_1}\orth \parr X_3}
\end{prooftree*}

Remarquons que l'utilisateur n'a eu besoin de donner que deux informations, et que tout le reste de la preuve en a été déduit. Cela a été permis par la compacité de nos représentations, et par la boucle d'autocomplétion.

\section{Démonstrations annexes}

\begin{demonstrationappendix}
    \label{proof_img_encode}
    On démontre le théorème~\ref{encode_img} par induction sur $\someproof \in \proofs$:
    \begin{description}
    \item[Axiome:] $\someproof =
    \begin{prooftree}
        \axv{X\orth, X}
    \end{prooftree}$,
    et donc $\encode \left( \someproof \right) = F(1 \epsilon, 2 \epsilon) \vdash X\orth, X$

    On vérifie aisément les conditions \ref{cadd}, \ref{clin} et \ref{cdes} de la définition~\ref{def_rep}, d'où $\encode \left( \someproof \right) \in \representations$.

    \item[Échange:] $\someproof =
    \begin{prooftree}
        \namedproofv{\pi_1}{\sequent}
        \permv{\someperm}{\permapp{\someperm}{\sequent}}
    \end{prooftree}$.
    
    Par hypothèse d'induction, $\encode \left( \pi_1 \right) = \encode ' \left( \pi_1 \right) \vdash \sequent$ vérifie \ref{cadd}, \ref{clin} et \ref{cdes}. Les conditions \ref{clin} et \ref{cdes} sont stables par ré-adressage par $\someperm$, étant donné que les conditions portent sur les adresses de l'arbre et du séquent, et qu'on ré-adresse les deux identiquement. La condition \ref{cadd} est également préservée, car on s'assure de trier les adresses des feuilles après avoir ré-adressé par $\someperm$.

    On en déduit que $\encode \left( \someproof \right) = \someperm' \left( \encode ' \left( \pi_1 \right) \right) \vdash \permapp{\someperm}{\sequent}$ vérifie \ref{cadd}, \ref{clin} et \ref{cdes}, d'où $\encode \left( \someproof \right) \in \representations$.

    \item[Par:] $\someproof =
    \begin{prooftree}
      \namedproofv{\pi_1}{\sequent, A, B, \sequentbis}
      \parrv{\sequent, A \parr B, \sequentbis}
    \end{prooftree}$

    Par hypothèse d'induction, $\encode \left( \pi_1 \right) = \encode ' \left( \pi_1 \right) \vdash \sequent, A, B, \sequentbis$ vérifie \ref{cadd}, \ref{clin} et \ref{cdes}.
    
    On a, de plus:
    \begin{equation*}
    \encode ( \someproof ) = \begin{tikzpicture}%
    [level 2/.style={sibling distance=3.5cm}]
    \node {$U_{n \epsilon}$}
        child {node {$\psi_\parr \left( \encode' \left( \pi_1 \right) \right)$}
    };
    \end{tikzpicture} \vdash \sequent, A \parr B, \sequentbis
    \end{equation*}

    Vérifions les trois conditions:
    \begin{enumerate}
        \item\label{pcadd} L'adresse $n \epsilon$, adressant un par, étiquette la racine d'arité 1, celle-ci est donc correcte.
        
        Par ailleurs, \ref{cadd} est vérifié pour $\encode ' \left( \pi_1 \right) \vdash \sequent, A, B, \sequentbis$, d'où, grâce au ré-adressage effectué par $\psi_\parr$, les adresses de $\psi_\parr \left( \encode' \left( \pi_1 \right) \right)$ sont correctes. En effet, les sous-adresses de $A$ deviennent des sous-adresses gauches de $A \parr B$, celles de $B$ des sous-adresses droites de $A \parr B$, et les adresses de $\sequentbis$ dont décalées de $1$, étant donné que $A$ et $B$ ont été fusionnées, diminuant la longueur du séquent de $1$.
        
        Toutes les adresses sont donc correctes, et le bon adressage est préservé.
        
        \item D'une part, \ref{clin} étant vérifiée pour $\encode ' \left( \pi_1 \right) \vdash \sequent, A, B, \sequentbis$, et le ré-adressage par $\psi_\parr$ étant injectif, les adresses de $\psi_\parr \left( \encode' \left( \pi_1 \right) \right)$ sont distinctes.
        
        D'autre part, chaque sous-formule de $\sequent, A, B, \sequentbis$ est étiquetée par au moins un noeud de $\encode ' \left( \pi_1 \right)$, d'après \ref{clin}. Le ré-adressage par $\psi_\parr$ étant correct (voir preuve de \ref{pcadd}), chaque sous-formule de $\sequent, A \parr B, \sequentbis$, à l'exception de $A \parr B$, est étiquetée par au moins un noeud de $\psi_\parr \left( \encode' \left( \pi_1 \right) \right)$.
        
        Chaque sous-formule de $\sequent, A, B, \sequentbis$ est donc étiquetée par un unique noeud de $\encode ' \left( \pi_1 \right)$.
        
        $A \parr B$ est quant à lui étiqueté par la racine, et c'est le seul endroit où son adresse apparaît, car, par construction de $\psi_\parr$, aucun noeud de $\psi_\parr \left( \encode' \left( \pi_1 \right) \right)$ ne peut être étiqueté par $n \epsilon$.

        Finalement, chaque sous-adresse de $\sequent, A \parr B, \sequentbis$ apparaît une et une seule fois dans l'arbre, la linéarité en est donc préservée.
        
        \item La condition \ref{cdes} est vérifiée pour $\encode ' \left( \pi_1 \right)$. Intéressons-nous au nouvel arbre.
        
        La racine est étiquetée par $n \epsilon$, ce qui convient.

        Par ailleurs, le ré-adressage par $\psi_\parr$ préserve le fait d'être une sous-adresse. On vérifie donc aisément que la condition \ref{cdes} est maintenue pour les adresses $(m, \someadd)$ de $\encode' \left( \pi_1 \right)$, où $m \notin \{ n, \; n+1 \}$.
        
        Si $m = n$: soit $\someadd = \epsilon$, auquel cas l'adresse modifiée vaut $(n, \Left)$ dans $\psi_\parr \left( \encode' \left( \pi_1 \right) \right)$, et descend de la racine étiquetée par $(n, \epsilon)$, soit $\someadd = \someadd' \cdot \alpha$, d'où, d'après \ref{cdes}, l'adresse $(n, \someadd)$ descend de $(n, \someadd')$ dans $\psi_\parr \left( \encode' \left( \pi_1 \right) \right)$, et l'adresse modifiée $(n, \Left \cdot \someadd)$ descend de $(n, \Left \cdot \someadd')$ dans $\psi_\parr \left( \encode' \left( \pi_1 \right) \right)$, ce qui convient.

        Le cas $m = n+1$ se traite de même.

        Ainsi, \ref{cdes} est vérifiée pour le nouvel arbre.
    \end{enumerate} 

    On a bien \ref{cadd}, \ref{clin} et \ref{cdes}, d'où $\encode \left( \someproof \right) \in \representations$.

     \item[Tenseur:] $\someproof =
    \begin{prooftree}
      \namedproofv{\pi_1}{\sequent, A}
      \namedproofv{\pi_2}{B, \sequentbis}
      \tensorv{\sequent, A \tensor B, \sequentbis}
    \end{prooftree}$

    Par hypothèses d'induction, $\encode \left( \pi_1 \right) = \encode ' \left( \pi_1 \right) \vdash \sequent, A$ et $\encode \left( \pi_2 \right) = \encode ' \left( \pi_2 \right) \vdash B, \sequentbis$ vérifient \ref{cadd}, \ref{clin} et \ref{cdes}.
    
    De plus:
    \begin{equation*}
    \encode ( \someproof ) = \begin{tikzpicture}%
    [level 1/.style={sibling distance=3.5cm}]
    \node {$B_{n \epsilon}$}
        child {node {$\psi_{\tensor\Left} \left( \encode' \left( \pi_1 \right) \right)$}}
        child {node {$\psi_{\tensor\Right} \left( \encode' \left( \pi_2 \right) \right)$}
    };
    \end{tikzpicture} \vdash \sequent, A \tensor B, \sequentbis
    \end{equation*}

    Vérifions les trois conditions:
    \begin{enumerate}
        \item L'argument est similaire au cas du par: les adresses de $A$ et $B$ deviennent des sous-adresses gauches et droites de $A \tensor B$, et un décalage est effectué sur les adresses de $\sequentbis$, à cause de la fusion des séquents.

        \item À nouveau, l'argument pour $\psi_{\tensor\Left} \left( \encode' \left( \pi_1 \right) \right)$ et $\psi_{\tensor\Right} \left( \encode' \left( \pi_2 \right) \right)$ est similaire au cas du par: le ré-adressage est correct, et on se sert des hypothèses d'induction.
        
        Le cas de la racine se traite également comme pour le par.

        \item De la même manière que pour le par, en utilisant les hypothèses d'induction et par construction des fonctions de ré-adressage, la descendance est préservée au sein de $\psi_{\tensor\Left} \left( \encode' \left( \pi_1 \right) \right)$ et $\psi_{\tensor\Right} \left( \encode' \left( \pi_2 \right) \right)$.

        Par ailleurs, la racine est étiquetée par $n \epsilon$, ce qui convient. Il s'agit de plus d'un noeud de la forme $B(n \epsilon)$, et, par construction de $\psi_{\tensor\Left}$, aucun noeud de $\psi_{\tensor\Left} \left( \encode' \left( \pi_1 \right) \right)$ ne peut être étiqueté par $n \Right$. Idem pour $\psi_{\tensor\Right} \left( \encode' \left( \pi_2 \right) \right)$ qui ne peut contenir $n \Left$.

        La racine vérifie ainsi toutes les conditions, la descendance est donc préservée.
    \end{enumerate}

    On en déduit $\encode \left( \someproof \right) \in \representations$.
    \end{description}

    Ainsi, $\forall \someproof \in \proofs, \encode \left( \someproof \right) \in \representations$.
    \qed{}
\end{demonstrationappendix}

\begin{demonstrationappendix}
    \label{proof_rep_inversion}
    On démontre le théorème~\ref{rep_inversion} par récurrence sur la hauteur $\height$ de $t \in \trees$. On regarde le n\oe ud présent à la racine:
    \begin{description}
        \item[$F$:]    
        Si $(t, \sequent) = F(n\rho, n'\rho') \vdash \sequent \in \representations$.
    
        D'après la condition \ref{cdes} de descendance, on a nécessairement $\rho = \rho' = \epsilon$.
    
        D'après la condition \ref{clin} de linéarité, $n \neq n'$, $\size{\sequent} = 2$ et $\{n, n'\} = \{1, 2\}$.
    
        D'après la condition \ref{cadd} de bon adressage, $n = 1$, $n' = 2$, et $\sequent = X\orth, X$ ou $\sequent = X, X\orth$, avec $X$ un atome.
    
        Dans le premier cas,
        $\someproof = \begin{prooftree}
                \axv{X\orth, X}
            \end{prooftree}$ convient donc, puisque $\encode \left( \someproof \right) = F(1 \epsilon, 2 \epsilon) \vdash X\orth, X = (t, \sequent)$.
    
        Dans le second cas,
        $\someproof = \begin{prooftree}
                \axv{X\orth, X}
                \permv{(2, 1)}{X, X\orth}
            \end{prooftree}$ convient donc, puisque $\encode \left( \someproof \right) = F(1 \epsilon, 2 \epsilon) \vdash X, X\orth = (t, \sequent)$.
         
        \item[$U$:] Si $(t, \sequent) = U(n\rho, t') \vdash \sequent \in \representations$.
    
        D'après la condition \ref{cdes}, $\rho = \epsilon$.
    
        D'après la condition \ref{cadd}, $\sequent = \sequent_1, \; F_1 \parr F_2, \; \sequent_2$, et $n\epsilon$ adresse $F_1 \parr F_2$ ($\size{\sequent_1} = n-1$). Puis, d'après \ref{cadd} et \ref{clin}, les adresses de $t'$ sont exactement celles des sous-formules de $\sequent$, à l'exception de $F_1 \parr F_2$ $\bigstar$.
    
        Modifions alors $t'$ pour obtenir une représentation correcte plus petite. On pose 
        \begin{equation*}
            \psi_\parr^{-1} =
            \begin{cases*}
                (i, \someadd) \mapsto (i, \someadd) & si $i < n$\\
                (n, \Left \cdot \someadd) \mapsto (n, \someadd)\\
                (n, \Right \cdot \someadd) \mapsto (n+1, \someadd)\\
                (i, \someadd) \mapsto (i+1, \someadd) & si $i \geq n+1$
            \end{cases*}
        \end{equation*}
        
        et on montre que $\left( \psi_\parr^{-1} \left( t' \right), \; \sequent_1, F_1, F_2, \sequent_2 \right)$ vérifie \ref{cadd}, \ref{clin}, \ref{cdes}. Remarquons d'abord que la définition de $\psi_\parr^{-1}$ ne pose pas problème, puisque, d'après \ref{clin}, les adresses de $t$ sont distinctes, $n \epsilon$ n'apparaît donc pas dans $t'$.
    
        \begin{enumerate}
            \item Cela découle de $\bigstar$, et de la correction du ré-adressage effectué par $\psi_\parr^{-1}$ (la preuve est similaire à celles faites dans la preuve du théorème~\ref{encode_img}).
    
            \item Idem.
    
            \item La descendance est conservée par $\psi_\parr^{-1}$ (à nouveau, la preuve est similaire à celles du théorème~\ref{encode_img}).
        \end{enumerate}
    
        Ainsi, $\left( \psi_\parr^{-1} \left( t' \right), \; \sequent_1, F_1, F_2, \sequent_2 \right) \in \representations$. Par hypothèse de récurrence, il existe donc $\pi' \in \proofs$ tel que $\encode \left( \pi' \right) = \left( \psi_\parr^{-1} \left( t' \right), \; \sequent_1, F_1, F_2, \sequent_2 \right)$.
        
        D'après la définition de $\encode$, $\pi'$ a nécessairement pour conclusion $\sequent_1, F_1, F_2, \sequent_2$, d'où
        $\pi = 
        \begin{prooftree}
            \namedproofv{\pi'}{\sequent_1, F_1, F_2, \sequent_2}
            \parrv{\sequent_1, F_1 \parr F_2, \sequent_2}
        \end{prooftree} 
        \in \proofs$
    
        De plus, $\encode \left( \pi \right) = 
        \begin{tikzpicture}%
        [level 2/.style={sibling distance=3.5cm}]
        \node {$U_{n \epsilon}$}
            child {node {$\psi_\parr \left( \psi_\parr^{-1} \left(
               t' \right) \right)$}
        };
        \end{tikzpicture} \vdash \sequent_1, F_1 \parr F_2, \sequent_2$.
    
        On vérifie bien, par définition, que $\psi_\parr \circ \psi_\parr^{-1} = \textit{Id}$, d'où $\encode \left( \pi \right) = (t, \sequent)$.
    
        \item[$B$:] Si $(t, \sequent) = B(n\rho, t_1, t_2) \vdash \sequent \in \representations$.
    
        D'après la condition \ref{cdes}, $\rho = \epsilon$.
    
        D'après la condition \ref{cadd}, $\sequent = \sequent_1, \; F_1 \tensor F_2, \; \sequent_2$, et $n\epsilon$ adresse $F_1 \tensor F_2$. 
    
        De plus, par \ref{clin}, toutes les sous-adresses de $F_1$ apparaissent exactement une fois dans $t$. D'après \ref{cdes}, elles apparaissent nécessairement dans $t_1$. Idem pour les sous-adresses de $F_2$, qui apparaissent exactement une fois dans $t_2$.
    
        Par ailleurs, en notant $\sequent' = \sequent_1, \sequent_2$, d'après \ref{clin}, il existe deux uniques séquents $G$ et $D$ tels que $\sequent' = G \setshuffle  D$ (``shuffle'': $G$ et $D$ forment une paritition des éléments de $\sequent'$ respectant chacun l'ordre des éléments dans $\sequent'$), où les formules de $G$ sont adressées dans $\sequent$ par des adresses de $t_1$, et celles de $D$ par des adresses de $t_2$. En effet, chaque formule de $\sequent'$ apparaît exactement une fois dans $t$, et chaque formule apparaît nécessairement dans $t_1$ ou $t_2$, car la racine adresse $F_1 \tensor F_2$.
    
        Posons alors $\sequentbis_1 = G, F_1$, et $\sequentbis_2 = F_2, D$. D'après les propriétés déjà énoncées, les formules de $\sequentbis_1$ sont des formules de $\sequent$ adressées par $t_1$, et celles de $\sequentbis_2$ des formules de $\sequent$ adressées par $t_2$. De plus, par construction, les sous-formules de $\sequent$ correspondent exactement à celles de $\sequentbis_1$ et $\sequentbis_2$, en leur ajoutant $F_1 \tensor F_2$ $\bigstar_1$.
        
        D'après \ref{cadd} et \ref{clin}, les adresses de $t$ sont exactement celles des sous-formules de $\sequent$, et, par ailleurs, $F_1 \tensor F_2$ est adressé par la racine. On en déduit que $\sequentbis_1$ correspond exactement aux formules de $\sequent$ dont les sous-adresses sont dans $t_1$, idem pour $\sequentbis_2$, dont les sous-adresses sont exactement celles de $t_2$ $\bigstar_2$.
    
        Construisons désormais des preuves de $\sequentbis_1$ et $\sequentbis_2$. Soient
        \begin{align*}
            \psi_1 &=
            \begin{cases*}
                (i, \someadd) \mapsto (\beta_{\sequentbis_1} \left( i \right), \someadd) & si $i \neq n$\\
                (n, \Left \cdot \someadd) \mapsto (| \sequentbis_1 | + 1, \someadd)
            \end{cases*} \\[1ex]
            \psi_2 &=
            \begin{cases*}
                (i, \someadd) \mapsto (\beta_{\sequentbis_2} \left( i \right) + 1, \someadd) & si $i \neq n$\\
                (n, \Right \cdot \someadd) \mapsto (1, \someadd)
            \end{cases*} 
        \end{align*}
    
        où $\beta_{\sequentbis_j} \left( i \right)$ est la position de la $i\ieme$ formule de $\sequent$ dans $\sequentbis_j$. Par exemple, si $\sequent = A, B, C, D$, et $\sequentbis_1 = B, C$, on a $\beta_{\sequentbis_1} \left( 3 \right) = 2$ (position de $C$).
    
        On vérifie alors que $\left( \psi_1 \left( t_1 \right), \; \sequentbis_1 \right)$ et $\left( \psi_2 \left( t_2 \right), \; \sequentbis_2 \right)$ sont des représentations correctes. À nouveau, d'après la condition \ref{cdes} et la propriété $\bigstar_2$, les définitions de $\psi_1$ et $\psi_2$ ne posent pas problème. Détaillons les trois conditions pour $\left( \psi_1 \left( t_1 \right), \; \sequentbis_1 \right)$:
    
        \begin{enumerate}
            \item Cela découle de $\bigstar_2$ et du ré-adressage effectué par $\psi_1$, dont on vérifie aisément la correction.
            
            \item Idem.
    
            \item Soit $(m, \someadd)$ une adresse de $t_1$.
            
            Si $m \neq n$, la descendance est clairement préservée par $\psi_1$.
            
            Si $m = n$: $\someadd$ ne peut valoir $\epsilon$, car $(n, \epsilon)$ étiquette la racine de $t$. On a donc soit $\someadd = \Left$, soit $\someadd = \Left \cdot \someadd' \cdot \alpha$, d'après la condition \ref{cdes} sur $t$ (la racine étant un tenseur). Dans le premier cas, l'adresse devient $(| \sequentbis_1 |, \epsilon)$. Dans le second, $(n, \someadd)$ descend de $(n, \Left \cdot \someadd' )$ dans $t_1$, d'après \ref{cdes}, d'où l'adresse modifiée $(| \sequentbis_1 |, \someadd' \cdot \alpha)$ descend de $(| \sequentbis_1 |, \someadd')$ dans $\psi_1 \left( t_1 \right)$, ce qui convient.
        \end{enumerate}
    
        Les trois conditions se prouvent de même pour $\left( \psi_2 \left( t_2 \right), \; \sequentbis_2 \right)$.
    
        Par hypothèse de récurrence, il existe donc $\pi_1, \pi_2 \in \proofs$ telles que $\encode \left( \pi_1 \right) = \left( \psi_1 \left( t_1 \right), \; \sequentbis_1 \right)$ et $\encode \left( \pi_2 \right) = \left( \psi_2 \left( t_2 \right), \; \sequentbis_2 \right)$. D'après la définition de $\encode$, $\pi_1$ a nécessairement $\sequentbis_1$ pour conclusion, et $\pi_2$ a pour conclusion $\sequentbis_2$.
        
        On en déduit que 
        $\begin{prooftree}
            \namedproofv{\pi_1}{\sequentbis_1 = G, F_1}
            \namedproofv{\pi_2}{\sequentbis_1 = F_2, D}
            \tensorv{G, F_1 \tensor F_2, D}
        \end{prooftree} 
        \in \proofs$. 
    
        Par ailleurs, d'après $\bigstar_2$, il existe une unique permutation $\someperm$ telle que $\permapp{\someperm}{(G, F_1 \tensor F_2, D)} = \sequent_1, F_1 \tensor F_2, \sequent_2$. Posons $n' = | G | + 1$, et exprimons celle-ci explicitement :
        \begin{equation*}
            \someperm = 
            \begin{cases*}
                i \mapsto \beta_{\sequentbis_1}^{-1} \left( i \right) & si $i < n' $\\
                n' \mapsto n \\
                i \mapsto \beta_{\sequentbis_2}^{-1} \left( i - n' + 1 \right) & si $i > n' $\\
            \end{cases*}
        \end{equation*}
    
        Ainsi, $\pi =
        \begin{prooftree}
            \namedproofv{\pi_1}{\sequentbis_1 = G, F_1}
            \namedproofv{\pi_2}{\sequentbis_1 = F_2, D}
            \tensorv{G, F_1 \tensor F_2, D}
            \permv{\someperm}{\sequent_1, F_1 \tensor F_2, \sequent_2}
        \end{prooftree} 
        \in \proofs$. 
    
        De plus, 
        \begin{equation*}
            \begin{split}
            \encode' \left( \pi \right) & = 
            \someperm \left( \encode' \left( 
            \begin{prooftree}
                \namedproofv{\pi_1}{\sequentbis_1 = G, F_1}
                \namedproofv{\pi_2}{\sequentbis_1 = F_2, D}
                \tensorv{G, F_1 \tensor F_2, D}
            \end{prooftree}
            \right) \right) \\
            & = \someperm \left(
            \begin{tikzpicture}%
            [level 1/.style={sibling distance=3.5cm}]
            \node {$B_{n' \epsilon}$}
                child {node {$\psi_{\tensor\Left} \left( \encode' \left(
                        \pi_1
                      \right) \right)$}}
                child {node {$\psi_{\tensor\Right} \left( \encode' \left(
                        \pi_2
                      \right) \right)$}
            };
            \end{tikzpicture} 
            \right) \\
            & = \begin{tikzpicture}%
            [level 1/.style={sibling distance=3.5cm}]
            \node {$B_{n \epsilon}$}
                child {node {$\someperm \left( \psi_{\tensor\Left} \left( \psi_1 \left(
                        t_1
                      \right) \right) \right)$}}
                child {node {$\someperm \left( \psi_{\tensor\Right} \left( \psi_2 \left(
                        t_2
                      \right) \right) \right)$}
            };
            \end{tikzpicture} 
            \end{split}
        \end{equation*}
    
        On vérifie, d'après les définitions, que $\someperm \circ \psi_{\tensor\Left} \circ \psi_1 = \textit{Id}$, et $\someperm \circ \psi_{\tensor\Right} \circ \psi_2 = \textit{Id}$.
    
        Il en découle finalement que $\encode \left( \pi \right) = 
        \begin{tikzpicture}%
            [level 1/.style={sibling distance=3.5cm}]
            \node {$B_{n \epsilon}$}
                child {node {$t_1$}}
                child {node {$t_2$}
            };
            \end{tikzpicture} 
            \vdash \sequent_1, F_1 \tensor F_2, \sequent_2$. 
            
            Soit $\encode \left( \pi \right) = \left( t, \sequent \right)$.
\qed
    \end{description}
\end{demonstrationappendix}

\begin{demonstrationappendix}
    \label{highapprox_correction_proof}
    Montrons l'invariant de la démonstration du théorème~\ref{highapprox_correction} par récurrence sur le nombre d'appels.

    \begin{description}
        \item[Initialisation:]
            Lors du premier appel, on a $t = t_1$ et $a'_1 = a'$, d'où $\alpha_1 = \epsilon$. Par ailleurs, $\highapprox = \Sigma$. 
    
            De manière évidente, $t_1$ est sous-arbre de $t$, et $a'_1$ adresse dans $t_1$ le trou adressé par $a'$ dans $t$.
            
            On vérifie de plus que $\highapprox = \highapproxspec \left( \treesimplify( t, \; \epsilon ), \sequent, \; \epsilon \right) = \highapproxspec \left( H, \sequent, \; \epsilon \right)$: aucune adresse n'apparaît dans $\unknown$, et $\Sigma$ est bien l'ensemble des adresses racines correctes de $(\unknown, \sequent)$.

        \item[Hérédité:] Supposons que l'invariant est vérifié au début du $i$\ieme{} appel. On regarde le n\oe ud présent à la racine de $t_i$:

        \begin{description}
            \item[$F$]
                Ce cas est impossible, car $t_i$ ne contiendrait alors pas de trou.
                
            \item[$H$]
                Si $t_i = H$, il n'y a pas de $i+1$\ieme{} appel, on n'a donc rien à vérifier. 

            \item[$B$] Si $t_i = B(n\rho, t', t'')$.

            $a_i'$ adresse dans $t_i$ le trou adressé par $a'$ dans $t$, d'où $a_i'$ est nécessairement de la forme $\alpha \cdot a_{i+1}'$. Détaillons le cas où $\alpha = \Left$.

            On a, d'après le pseudo-code, $t_{i+1} = t'$, et $\alpha_{i+1} = \alpha_i \cdot \Left$. Ainsi, $t_{i+1}$ est bien sous-arbre de $t_i$, et, par hypothèse de récurrence, $\alpha_i$ adresse la racine de $t_i$, d'où $\alpha_{i+1}$ adresse la racine de $t_{i+1}$. De même, $a_{i+1}'$ adresse dans $t_{i+1}$ le trou adressé par $a_i'$ dans $t_i$, qui est par HR le trou adressé par $a'$ dans $t$.
            
            Par ailleurs, d'après le pseudo-code, ${\highapprox}^{i+1} = ({\highapprox}^{i}\setminus (\mathcal{M} \cup \{ (m, \someadd) \})) \cup \left\{ (m, \someadd \cdot \Left) \right\}$, où $\mathcal{M}$ est l'ensemble des adresses minimales de $t''$.

            Montrons que ${\highapprox}^{i+1} = \highapproxspec \left( \treesimplify( t, \; {\alpha}_{i+1} ), \sequent, \; {\alpha}_{i+1} \right)$:

            \begin{description}
                \item[$\subseteq$] 
                    Comme $(t, \sequent) \in \representationspartial$, on a par propriété sur la simplification que $(s(t, \alpha_{i}), \sequent) \in \representationspartial$. Soit $\beta = (m, \someadd) \in {\highapprox}^{i+1}$. On a deux cas:

                    \begin{description}
                        \item[Si $m = n$:] 
                            Nécessairement, $\beta = (n, \rho \cdot \Left)$, car, par hypothèse de récurrence, ${\highapprox}^{i}$ était correct, et, comme $(s(t, \alpha_{i}), \sequent) \in \representationspartial$, cet ensemble ne contenait qu'une seule adresse d'entier $n$, à savoir $(n, \rho)$. Cela vient du fait que $(n, \rho)$ étiquette la racine de $t_i$, et donc que $(n, \rho)$ descend immédiatement d'une adresse sur la branche racine--trou de $s(t, \alpha_{i})$.
                            
                            Ainsi, $\beta = (n, \rho \cdot \Left)$ descend immédiatement d'une adresse de $s(t_{i+1}, \alpha_{i+1})$, en respectant la condition gauche de descendance. De plus, c'est une adresse correcte, car $(t, \sequent) \in \representationspartial$, et cet arbre contient un n\oe ud $B(n \rho, t', t'')$ (la racine de $t_i$). On en déduit que $(n \rho)$ adresse un tenseur, qui a donc bien un fils gauche. 
                            
                            L'adresse $(n, \rho \cdot \Left)$ ne peut enfin pas apparaître dans $s(t, \alpha_{i+1})$. En effet, elle n'apparaît pas dans $t''$ à cause de la condition de descendance \ref{clin}, et elle n'apparaît pas non plus dans le reste de l'arbre car elle n'apparaissait déjà pas dans $s(t, \alpha_{i})$, par HR (car $(n, \rho)$ n'y apparaissait pas).

                            Finalement, $\beta \in \highapproxspec \left( \treesimplify( t, \; {\alpha}_{i+1} ), \sequent, \; {\alpha}_{i+1} \right)$.
                            
                        \item[Si $m \neq n$:] 
                            Nécessairement, $\beta \in {\highapprox}^{i}$, et, de plus, $\beta \notin \mathcal{M}$. Le fait que $\beta \in {\highapprox}^{i}$ assure qu'il s'agit d'une adresse racine correcte, ou d'une adresse correcte descendant immédiatement d'une adresse de la branche racine--trou de $s(t, \alpha_{i})$, et n'y apparaisssant pas. Comme $m \neq n$ et $\beta \notin \mathcal{M}$, ces conditions sont toujours vérifiées pour $s(t, \alpha_{i+1})$.
                        
                        D'où $\beta \in \highapproxspec \left( \treesimplify( t, \; {\alpha}_{i+1} ), \sequent, \; {\alpha}_{i+1} \right)$, et donc ${\highapprox}^{i+1} \subseteq \highapproxspec \left( \treesimplify( t, \; {\alpha}_{i+1} ), \sequent, \; {\alpha}_{i+1} \right)$.
                    \end{description}

                \item[$\supseteq$]
                    Supposons par l'absurde qu'il existe $\beta \in \highapproxspec \left( \treesimplify( t_{i+1}, \; {\alpha}_{i+1} ), \sequent, \; {\alpha}_{i+1} \right)$, telle que $\beta \notin {\highapprox}^{i+1}$. 
                    
                    Nécessairement, $\beta \neq (n, \someadd \cdot \Left)$, car $\beta \notin {\highapprox}^{i+1}$. On en déduit, par un argument de descendance, que $\beta = (m, \someadd')$ avec $m \neq n$. En utilisant un argument similaire à la disjonction du cas $\subseteq$, il en découle que $\beta \in \highapproxspec \left( \treesimplify( t_{i}, \; {\alpha}_{i} ) \right)$. Par hypothèse de récurrence, ${\highapprox}^{i}$ est correct, d'où $\beta \in {\highapprox}^{i}$.
                    
                    De plus, $\beta$ n'apparaît pas dans $s(t, \alpha_{i+1})$, d'où en particulier $\beta \notin \mathcal{M}$, et donc $\beta \notin \mathcal{M} \cup \{ (n, \someadd) \}$.

                    Par construction de ${\highapprox}^{i+1}$, on devrait alors avoir $\beta \in {\highapprox}^{i+1}$, ce qui est absurde.
            \end{description}

            On a bien ${\highapprox}^{i+1} = \highapproxspec \left( \treesimplify( t, \; {\alpha}_{i+1} ), \sequent, \; {\alpha}_{i+1} \right)$, l'invariant est conservé.
    
            \item[$U$] Si $t_i = U(n\rho, t') \in \representations$.
    
            $a_i'$ adresse dans $t_i$ le trou adressé par $a'$ dans $t$, d'où, nécessairement, $a_i' = \Left \cdot a_{i+1}'$ (d'après la convention d'adressage dans un arbre). Ce cas correspond bien à celui du pseudo-code, et l'argument de correction est similaire au cas du constructeur $B$.
\qed
        \end{description}
    \end{description}
\end{demonstrationappendix}

\begin{theorem}[Propriétés de contexte]
    \label{context_prop}
    Soient $(t_1, \sequentbis_1), \;  (t_2, \sequentbis_2) \in \representationspartial$. 
    
    Supposons que $\sequentbis_1 = G, F_1$, et $\sequentbis_2 = F_2, D$, avec $F_1$, $F_2$ des formules, et $G$, $D$ des séquents. On suppose de plus qu'il existe une permutation $\someperm$ et un séquent $\sequent$ tels que $\permapp{\someperm}{(G, F_1 \tensor F_2, D)} = \sequent_1, F_1 \tensor F_2, \sequent_2$.

    On s'intéresse aux représentations partielles ``contextuelles'' suivantes:

    \begin{align*}
        \begin{multlined}
            t'_1 = \begin{tikzpicture}%
            [level 1/.style={sibling distance=3.5cm}]
            \node {$B_{n \epsilon}$}
                child {node {$\someperm \left( \psi_{\tensor\Left} \left(
                t_1\right) \right)$}}
                child {node {$\unknown$}
            };
            \end{tikzpicture}
            \vdash \sequent \\
            t'_2 = \begin{tikzpicture}%
            [level 1/.style={sibling distance=3.5cm}]
            \node {$B_{n \epsilon}$}
                child {node {$\gamma$}}
                child {node {$\someperm \left( \psi_{\tensor\Right} \left(t_2\right) \right) = \unknown$}
            };
            \end{tikzpicture}
            \vdash \sequent \text{, où $\gamma$ est un arbre quelconque de $\trees$ }.
        \end{multlined}
    \end{align*}
    
    Si $a$ adresse un trou dans $t_1$, alors $a' = \Left \cdot a$ adresse le même trou dans $t'_1$, et on a $\highapprox(t_1, \sequentbis_1, a') \subseteq \Left \cdot \someperm \left( \psi_{\tensor\Left} \left(\highapprox(t'_1, \sequent, a')\right) \right)$. 
    
    De plus, $\Left \cdot \someperm \left( \psi_{\tensor\Left} \left(\lowapprox(t'_1, \sequent, a')\right) \right) \subseteq \lowapprox(t_1, \sequentbis_1, a')$.
    
    Des propriétés d'inclusion similaires s'énoncent pour $(t'_2, \sequent)$.
\end{theorem}

\begin{remark}
    La notation $\Left \cdot \Sigma$ signifie: les adresses de l'ensemble $\Sigma$ qu'on a préfixées par $\Left$.
\end{remark}

\begin{remark}
    Il découle de ces propriétés que la suite $(t'_i)$ construite dans la démonstration du théorème~\ref{completeness} est exacte selon $\sequent$:

    Supposons par l'absurde qu'elle ne le soit pas, et supposons de plus, sans perte de généralité, qu'une des deux conditions d'exactitude soit violée dans la première moitié de la suite utilisant $(g_i)$ (l'argument serait similaire pour la moitié de la suite utilisant $(d_i)$):

    \begin{itemize}
        \item[Si la condition $\exactcond$ était violée:] On aurait remplacé un trou d'un arbre 
            \begin{equation*}
            \begin{tikzpicture}%
            [level 1/.style={sibling distance=3.5cm}]
            \node {$B_{n \epsilon}$}
                child {node {$\someperm \left( \psi_{\tensor\Left} \left(
                g_i\right) \right)$}}
                child {node {$\unknown$}
            };
            \end{tikzpicture}
            \end{equation*}
            par une adresse n'appartenant pas à $\highapprox(t'_1, \sequent, \beta)$, où $\beta$ adresserait le trou en question. 
            
            Or, on aurait $\beta = \Left \cdot \beta'$, et, d'après les propriétés de contexte, $\highapprox(t_1, \sequentbis_1, \beta') \subseteq \Left \cdot \someperm \left( \psi_{\tensor\Left} \left(\highapprox(t'_1, \sequent, \beta)\right) \right)$. Cela signifierait que la condition $\exactcond$ aurait été également violée pour $g_i$, puisque l'adresse utilisée pour remplacer le trou de $g_i$ n'appartiendrait pas non plus à l'ensemble $\highapprox$ correspondant. 
            
            C'est absurde, car on a supposé que $(g_i)$ était exacte.

        \item[Si la condition $\exactcondbis$ était violée:] On aurait fermé une branche d'un arbre 
            \begin{equation*}
            \begin{tikzpicture}%
            [level 1/.style={sibling distance=3.5cm}]
            \node {$B_{n \epsilon}$}
                child {node {$\someperm \left( \psi_{\tensor\Left} \left(
                g_i\right) \right)$}}
                child {node {$\unknown$}
            };
            \end{tikzpicture}
            \end{equation*}
            par une feuille n'utilisant pas toutes les adresses de $\lowapprox(t'_1, \sequent, \beta)$, où $\beta$ adresserait le trou en question. 
            
            Or, on aurait $\beta = \Left \cdot \beta'$, et, d'après les propriétés de contexte, $\Left \cdot \someperm \left( \psi_{\tensor\Left} \left(\lowapprox(t'_1, \sequent, \beta)\right) \right) \subseteq \lowapprox(t_1, \sequentbis_1, \beta')$. À nouveau, cela signifierait que la condition $\exactcondbis$ aurait été violée pour $g_i$, car il existerait une adresse, appartenant à l'ensemble $\lowapprox$ correspondant, non utilisée pour fermer la branche de $g_i$. 
            
            C'est encore absurde, puisqu'on a supposé que $(g_i)$ était exacte.
    \end{itemize}

    Finalement, $(t'_i)$ est bien exacte selon $\sequent$. \qed
\end{remark}

\begin{proof}
    Ces propriétés se prouveraient de la même manière que ce qui a été fait dans les démonstrations annexes~\ref{highapprox_correction_proof} et~\ref{lowapprox_correction_proof}. Donnons les grandes lignes de la démonstration qu'on ferait pour les propriétés sur $t_1$ et $t'_1$.
    
    On considérerait des exécutions parallèles de l'algorithme sur $t'_1$ et $t_1$, et on montrerait à nouveau un invariant, plus général que le théorème, justifiant les propriétés de contexte pour des simplifications de $t'_1$ et $t_1$ liées aux appels récursifs. En effet, lors des appels, on descend le long des deux branches racine--trou : si l'on considère alors, à chaque étape, les n\oe uds sur lesquels on se trouve comme des trous, et qu'on simplifie $t'_1$ et $t_1$ en conséquence, on peut montrer, par invariant, les propriétés de contexte sur ces simplifications.

    Ces propriétés se propagent donc, et on en déduit, avec la terminaison de l'algorithme, les propriétés de contexte pour $t'_1$ et $t_1$
    
    La même idée serait utilisée pour les propriétés sur $t'_2$ et $t_2$.
\end{proof}

\end{document}
